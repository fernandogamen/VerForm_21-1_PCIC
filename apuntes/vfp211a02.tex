\documentclass[11pt,letterpaper]{article}
\usepackage{packageslc}
\usepackage{optionslc}

\usepackage{lstcoq}
\lstset{language=Haskell}
\newcommand{\Q}{\ensuremath{\mathbb{Q}}}
\newcommand{\Z}{\ensuremath{\mathbb{Z}}}
\newcommand{\N}{\ensuremath{\mathbb{N}}}
\newcommand{\R}{\ensuremath{\mathbb{R}}}
%\newtheorem{lemma}{Lema}[section]
%\newtheorem{theorem}[lemma]{Teorema}
%\newtheorem{corollary}[lemma]{Corolario}
%\theoremstyle{definition}
%\newtheorem{definition}[lemma]{Definici\'on}
\newcommand{\A}{\mathcal{A}}
\newcommand{\E}{\ensuremath{\exists}}
\newcommand{\iso}{\ensuremath{\cong}}
\newcommand{\union}{\ensuremath{\cup}}
\newcommand{\morinyec}{\ensuremath{\precapprox}}
\newenvironment{prueba}{\vspace{-5mm}\noindent\textbf{Demostraci\'on}\\}{\noindent$\blacksquare$\\}
\newcommand{\nin}{\ensuremath{\notin}}
\newcommand{\tog}{\makebox[7mm][l]}
\newcommand{\toge}{\makebox[11mm][l]}
\newcommand{\toget}{\makebox[13mm][l]}
\newcommand{\togeth}{\makebox[14mm][l]}
\newcommand{\togethe}{\makebox[15mm][l]}
\newcommand{\together}{\makebox[17mm][l]}
\newcommand{\niso}{\ensuremath{\not \cong}}
\renewcommand\contentsname{\'Indice}
%\renewcommand\chaptername{Cap\'itulo}
\renewcommand\indexname{\'Indice}

%%\newcommand{\qed}{\hfill$\mathbb{Qed}$}
%\newcommand{\qed}{\hfill$\mathsf{\boldsymbol{\dashv}}$}


\newcommand{\Ejercicios}{\section*{Ejercicios}}

 \newenvironment{manitas}{%
      \renewcommand{\labelitemi}{\ding{44}}%
      \vspace{-0.5cm}%
      \begin{itemize}%
      \setlength{\itemsep}{0pt}\setlength{\parsep}{0pt}\setlength{\topsep}{0pt}%
      }{\end{itemize}}
\newenvironment{malitos}{%
      \renewcommand{\labelitemi}%
            {\raisebox{1.5ex}{\makebox[0.3cm][l]{\begin{rotate}{-90}%
            \ding{43}\end{rotate}}}}%
      \vspace{-0.5cm}%
      \begin{itemize}%
      \setlength{\itemsep}{0pt}\setlength{\parsep}{0pt}\setlength{\topsep}{0pt}%
      }{\end{itemize}}
\newenvironment{ejercs}{
     \renewcommand{\labelenumi}{\thesection.\theenumi.-}
     \renewcommand{\labelenumii}{\theenumii)}
     \begin{enumerate}}
     {\end{enumerate}}

   \newcommand{\bej}{\begin{ejercs}}
\newcommand{\eej}{\end{ejercs}}
   
%%=================================================================================

\newcommand{\bc}{\begin{center}}
\newcommand{\ec}{\end{center}}
\newcommand{\be}{\begin{enumerate}}
\newcommand{\ee}{\end{enumerate}}
\newcommand{\bi}{\begin{itemize}}
\newcommand{\ei}{\end{itemize}}
\newcommand{\beq}{\begin{equation}}
\newcommand{\eeq}{\end{equation}}
\newcommand{\beqs}{\begin{equation*}}
\newcommand{\eeqs}{\end{equation*}}
\newcommand{\ba}{\begin{array}}
\newcommand{\ea}{\end{array}}

\newcommand{\imp}{\rightarrow}
\newcommand{\Imp}{\Rightarrow}
\renewcommand{\iff}{\leftrightarrow}
\newcommand{\Iff}{\Leftrightarrow}
\newcommand{\G}{\Gamma}
\newcommand{\D}{\Delta}
\newcommand{\F}{\mathcal{F}}
\newcommand{\Ge}{\mathcal{G}}
\newcommand{\Pe}{\mathcal{P}}
\newcommand{\I}{\mathcal{I}}
\newcommand{\C}{\mathcal{C}}
\newcommand{\K}{\mathcal{K}}
\newcommand{\Kb}{\mathbb{K}}
\newcommand{\Eb}{\mathbb{E}}
\newcommand{\Ebs}{\mathbb{E}^\star}
\newcommand{\Ob}{\mathbb{O}}
\newcommand{\Ib}{\mathbb{I}}
\newcommand{\kb}{\bbkappa}
\newcommand{\M}{\mathcal{M}}
\newcommand{\Nc}{\mathcal{N}}
%\newcommand{\E}{\mathcal{E}}
%\newcommand{\R}{\mathcal{R}}
%\newcommand{\Q}{\mathcal{Q}}
\newcommand{\Sc}{\mathcal{S}}
\newcommand{\Sf}{\mathsf{\Sigma}}
\newcommand{\Te}{\mathcal{T}}
\newcommand{\Rb}{\mathbb{R}}
\newcommand{\Qb}{\mathbb{Q}}
\newcommand{\Kbb}{\mathbb{K}}
\newcommand{\T}{\mathbb{\Theta}}
\renewcommand{\L}{\mathcal{L}}
\newcommand{\fa}{\forall}
\newcommand{\ex}{\exists}
\newcommand{\inc}{\subseteq}
\newcommand{\lb}{\lambda}
\newcommand{\al}{\alpha}
\newcommand{\ga}{\gamma}
\newcommand{\Db}{\mathbb{D}}
\newcommand{\Fb}{\mathbb{F}}
\newcommand{\De}{\mathcal{D}}

\newcommand{\mg}{\mathbb{m}}

\newcommand{\cg}{\mathbb{C}}
\newcommand{\dg}{\mathbb{D}}
\newcommand{\jg}{\mathbb{J}}
\newcommand{\Ha}{\mathcal{H}}
%\newcommand{\A}{\mathcal{A}}
\newcommand{\sg}{\mathbb{S}}

\def\stackunder#1#2{\mathrel{\mathop{#2}\limits_{#1}}}



%\renewcommand{\qed}{\hfill$\boldsymbol{\dashv}$}
%\newcommand{\qed}{\hfill$\mathbb{Qed}$}

\newcommand{\pt}[1]{\langle #1 \rangle}



\renewcommand{\S}{\mathbb{\Sigma}}


\newcommand{\doubt}{\Red{{\LARGE {\sf ??}}}}

\newcommand{\coment}[1]{\hfill\\ \Big[{\bf Comentario Privado:} #1\Big]}
\newcommand{\preg}[1]{\hfill\\ \BrickRed{{\bf Pregunta:} #1}}
\newcommand{\conjet}[1]{\hfill\\ \OliveGreen{{\bf Conjecura:} #1}}






\newcommand{\pendiente}{\BrickRed{{\sc Pendiente}}}
\newcommand{\verifpendiente}{\BrickRed{{\sc Verificación pendiente}}}


\newcommand{\Mg}{\mathbb{M}}
\newcommand{\Bg}{\mathbb{B}}
\newcommand{\Lg}{\mathbb{L}}
\newcommand{\Tg}{\mathbb{T}}

\newcommand{\sketch}{\Red{{\sc sketch}}}




\newcommand{\B}{\mathbb{B}}
%\newcommand{\N}{\mathbb{N}}




%\newenvironment{leterize}{%
%        \renewcommand{\theenumi}{\alph{enumi}}
%        \begin{enumerate}}{\end{enumerate}}

%\newenvironment{manitas}{%
%      \renewcommand{\labelitemi}{\ding{44}}%
%      \vspace{-0.5cm}%
%      \begin{itemize}%
%      \setlength{\itemsep}{0pt}\setlength{\parsep}{0pt}\setlength{\topsep}{0pt}%
%      }{\end{itemize}}



\newcommand{\Lb}{\Lambda} 



\newcommand{\Om}{\Omega}

\newcommand{\W}{\mathcal{W}}


\newcommand{\Bc}{\mathcal{B}}
\newcommand{\Df}{\mathfrak{D}}
\newcommand{\Dc}{\mathcal{D}}
%\newcommand{\Tc}{\mathcal{T}}
\newcommand{\Mf}{\mathfrak{M}}

\newcommand{\Sg}{\mathbb{S}}

\newtheorem{theorem}{Teorema}
\newcommand{\teo}[1]{\begin{theorem} #1 \end{theorem}}
\newtheorem{proposition}{Proposici\'on}
\newcommand{\prop}[1]{\begin{proposition} #1 \end{proposition}}
\newtheorem{definition}{Definici\'on}
\newcommand{\defin}[1]{\begin{definition} #1 \end{definition}}
\newtheorem{corollary}{Corolario}
\newcommand{\cor}[1]{\begin{corollary} #1 \end{corollary}}
\newtheorem{lemma}{Lema}
\newcommand{\lema}[1]{\begin{lemma} #1 \end{lemma}}
\newcommand{\dem}[1]{\begin{proof} #1 \end{proof}}

%\renewcommand{\qed}{\qedsymbol{$\mathbf{\dashv}$}}

\newcommand{\proof}{\hfill\\\noindent\textbf{\textit{Demostraci\'on. }}}






\newcommand{\restr}[2]{#1\!\!\boldsymbol{\restriction}\!#2}



\newcommand{\vacio}{\varnothing}
\newcommand{\done}{\ensuremath{\checkmark}}

\newcommand{\ida}{$\Rightarrow \; )$ }
\newcommand{\regr}{$\Leftarrow \; )$ }


\newcommand{\ol}[1]{\overline{#1}}


\newcommand{\Tsf}{\mathsf{T}}

\newcommand{\inds}[1]{\index[simb]{#1}}



%--------------------------------------------------------------------------

\DeclareMathAlphabet{\mathpzc}{OT1}{pzc}{m}{it}

%\newcommand{\case}{\mathsf{case}}
%\renewcommand\labelitemi{$\circ$}
%%\newcommand{\qed}{\hfill$\mathbb{Qed}$}
\newcommand{\qed}{\hfill$\mathsf{\boldsymbol{\dashv}}$}

%\newcommand{\id}{\mathsf{Id}}

%\newcommand{\uc}{\mathcal{U}}
%\newcommand{\Ic}{\mathcal{I}}
%\newcommand{\pc}{\mathcal{P}}
%\newcommand{\qc}{\mathcal{Q}}
%\newcommand{\mc}{\mathcal{M}}
\newcommand{\supc}{\supseteq}
\newcommand{\limo}{\mathop{\mathpzc{Lim}}}
\newcommand{\ord}{\mathsf{OR}}

\newcommand{\hint}{\emph{Sugerencia: }}


\newcounter{EjempCtr}[section]
\newenvironment{enumrom}{\renewcommand{\theenumi}{\roman{enumi}}%
\renewcommand{\theenumii}{\roman{enumii}}
\renewcommand{\theenumiii}{\roman{enumiii}}
\renewcommand{\theenumiv}{\roman{enumiv}}
\begin{enumerate}}{\end{enumerate}}
\newenvironment{Ejemplo}
        {\stepcounter{EjempCtr}%
        \begin{description}\item[Ejemplo \thesection.\arabic{EjempCtr}]}%
        {\end{description}}
\newenvironment{demostr}{%
             {\em Demostración:}
                \begin{quotation}}{\end{quotation}}

   \newcommand{\beje}{\begin{Ejemplo}}
\newcommand{\eeje}{\end{Ejemplo}}

\newcommand{\propo}{\ensuremath{\mathsf{PROP}}}
\newcommand{\atom}{\ensuremath{\mathsf{ATOM}}}
\newcommand{\vphi}{\varphi}
\newcommand{\vp}{\varphi}

\newcommand{\dn}{\mathsf{DN}}
\newcommand{\dnC}{\mathsf{DN_C}}
\newcommand{\dnM}{\mathsf{DN_M}}
\newcommand{\dnp}{\mathsf{DN_p}}
\newcommand{\dnm}{\mathsf{DN_p^M}}
\newcommand{\dnc}{\mathsf{DN_p^C}}

\newtheorem{eje}{Ejemplo}[section]
\newcommand{\ejem}[1]{\begin{eje}\normalfont #1 \end{eje}}

\newcommand{\vx}{\vec{x}}
\newcommand{\vy}{\vec{y}}
\newcommand{\vz}{\vec{z}}
\newcommand{\vt}{\vec{t}}
\newcommand{\vf}{\vec{f}}
\newcommand{\term}{\ensuremath{\mathsf{TERM}}}
\newcommand{\form}{\mathsf{FORM}}

\newcommand{\true}{\mathop{\mathsf{true}}}
\newcommand{\espc}{\vspace{.3cm}}
%%% Local Variables: 
%%% mode: latex
%%% TeX-master: t
%%% End: 

\newcommand{\scd}[2]{#1 \vdash #2}

\title{Verificaci\'on Formal PCIC 2021-2 \\
Un cálculo de secuentes
}
% \title{Lógica Computacional 2020-2, nota de clase 15 \\
% Un cálculo de secuentes
% }
\author{Favio Ezequiel Miranda Perea \and Araceli Liliana Reyes Cabello\and
Lourdes Del Carmen Gonz\'alez Huesca \and Pilar Selene Linares Arévalo}
\date{\today}

\begin{document}
\maketitle


% \section{Introducción}

% Los sistemas de deducción natural, introducidos por Gerhard Gentzen en 1935, son
% formalismos deductivos que modelan el razonamiento matemático ordinario de
% manera más fiel que un sistema axiomático o que el método de tableaux.  

% Un sistema de deducción natural consiste de reglas de inferencia donde las 
% hip\'otesis se encuentran en la parte superior de una l\'inea horizontal y la 
% conclusión en la parte inferior.
% Los sistemas que abordaremos en este curso se distinguen por que las reglas 
% describir\'an las formas para \textbf{introducir} y \textbf{eliminar} cada uno 
% de los conectivos lógicos. 
% Las pruebas o derivaciones se construyen en estos sistemas son mediante la 
% aplicación de dichas reglas en una sucesión adecuada que relaciona conclusiones 
% con premisas de reglas posteriores. De igual forma que en el razonamiento 
% ordinario se pueden hacer hipótesis temporales durante la prueba, las cuales se 
% pueden \textbf{descargar} al incorporarlas a la conclusión. 


% La importancia del uso de los sistemas de deducción natural son variados. 
% Los sistemas computacionales de razonamiento automatizado usualmente se basan 
% en métodos refutacionales como los tableaux pero los asistentes de prueba 
% interactivos son útiles para razonar acerca de las propiedades de programas y 
% est\'an basados en los sistemas lógicos de deducción natural.
% Adem\'as, en fundamentos de lenguajes de programación la deducción natural 
% juega también un papel importante por medio de la llamada correspondencia de 
% Curry-Howard también conocida como el paradigma de \textbf{fórmulas como 
% tipos}, 
% cuya idea a grandes rasgos es que las pruebas lógicas contienen ciertas 
% construcciones las cuales pueden interpretarse como programas, de modo que las 
% proposiciones lógicas se convierten en tipos de un lenguaje de programación. La 
% última parte de nuestro curso se dedicará en gran parte a mostrar tal 
% correspondencia.


% \subsection{Significado de conectivos y cuantificadores}

% El adjetivo \textbf{natural} dado por Gentzen a estos sistemas deductivos se
% refiere al hecho de que modelan de manera tan cercana como sea posible el
% razonamiento natural de un humano, o al menos el razonamiento
% matemático hecho por personas mediante distintos juicios. 
% Un juicio es una proposici\'on a cerca de una l\'ogica, en nuestro caso es a 
% cerca del significado de una fórmula lógica. Este significado se entenderá si 
% analizamos y comprendemos cu\'ando una f\'ormula es verdadera. 

% Analicemos a este respecto cada clase de fórmula de acuerdo a su 
% esquema, es decir de acuerdo a su conectivo o cuantificador principal y 
% tambi\'en tomando en cuenta su utilidad como generador u obtenci\'on de 
% informaci\'on:
% \bi
% \item Información básica: una f\'ormula~$ A$ es cierta o la proposici\'on 
% hecha 
% en $ A$ es verdadera, lo cual podemos denotar con $ A\,\true$. \\
% Este juicio ser\'a el m\'as b\'asico de todos. 
% \item Conjunción: la f\'ormula~$ A\land B$ es cierta solo sí ambas $ A$ y 
%   $ B$ son ciertas. Lo cual nos lleva al juicio:
%  \begin{mathpar}
%   \inferrule*[]{
%    A\,\true\and  B\,\true
%   }{
%    A\land B\,\true
%   }
%  \end{mathpar}
% Esta clase de regla se conoce como regla de introducción porque introduce un
% conectivo en la conclusión en este caso $\land$.

% La siguiente cuestión es preguntarnos c\'omo usar la información
% $ A\land B\,\true$, a cuya respuesta nos lleva de nuevo el razonamiento
% natural:
% \begin{mathpar}
%  \inferrule*[]{
%   A\land B\,\true
%  }{
%   A\,\true
%  }
 
%  \inferrule*[]{
%   A\land B\,\true
%  }{
%   B\,\true
%  }
% \end{mathpar}

% \item Implicación: ?`Cuando es verdadera una implicación ? El razonamiento
%   matemático nos dice que la implicación $ A\imp B$ es cierta si al
%   suponer el antecedente~$ A$ cierto podemos probar que el 
%   consecuente~$ B$ es cierto. Esto nos lleva a la siguiente regla de 
%   introducción:
%   \begin{mathpar}
%    \inferrule*[]{
%    [ A\;\true] \\\\
%    \vdots \\\\
%     B\;\true
%    }{
%     A\imp B\,\true
%    }
%   \end{mathpar}
% Aquí los corchetes que encierran al juicio hipotético $ A\,\true$ indican que 
% en la conclusión tal hipotésis fue descargada, es decir despu\'es de introducir 
% la implicación, la hipótesis hecha ya no es necesaria, es decir, se trataba de
% una hipótesis temporal.

% La regla de eliminación de la implicación modela una forma de razonamiento
% conocida desde Aristóteles y llamada \textit{modus ponens}. De los juicios
%  $ A\imp B\,\true$ y $ A\,\true$ podemos obtener el juicio $ B\,\true$:
%  \begin{mathpar}
%   \inferrule*[]{
%    A\imp B\,\true\and  A\,\true
%   }{
%    B\,\true
%   }
%  \end{mathpar}

%  \item La disyunción nos lleva a las siguientes reglas de introducción
%  \begin{mathpar}
%   \inferrule*[]{
%    A\,\true
%   }{
%    A\lor B\,\true
%   }
  
%   \inferrule*[]{
%    B\,\true
%   }{
%    A\lor B\,\true
%   }
%  \end{mathpar}
% Lo cual captura el hecho de que una disyunción es cierta sólo si alguna de
% sus dos componentes lo es. 

% Para obtener la regla de eliminación debemos considerar c\'omo utilizar 
% correctamente el juicio $ A\lor B\,\true$ dado que no sabemos con certeza 
% cu\'al de las dos componentes es cierta. 
% Si tratamos de probar $ C\,\true$ a partir de $ A\lor B\,\true$ debemos 
% llegar a tal juicio sin importar cu\'al de los dos casos se tome: $ A\,\true$ 
% o $ B\,\true$ válido. 
% Esto nos lleva a hacer una prueba por casos capturada en la siguiente regla:
% \begin{mathpar}
%  \inferrule*[]{
%  \qquad \qquad \qquad [ A\;\true] \qquad [ A\;\true] \\\\
%  \qquad \qquad \qquad \vdots \qquad \qquad \qquad\vdots \\\\
%  \vp\lor\psi\,\true \qquad  C\;\true \qquad  C\;\true 
%  }{
%   C\,\true
%  }
% \end{mathpar}
% Al igual que en la introducción de la implicación los corchetes indican que
% tal hipótesis es temporal y ha sido descargada.

% \item La falsedad~$\bot$ representa una contradicción y no debería ser 
%   probable, por lo que no tiene regla de introducción. \\
%   Inversamente si llegamos en algún momento al juicio~$\bot\,\true$ deberiamos 
%   poder concluir cualquier cosa, lo cual genera la regla de eliminación:
%   \begin{mathpar}
%    \inferrule*[]{
%    \bot\,\true
%    }{
%    \vp\,\true
%    }
%   \end{mathpar}

% \item La verdad debe ser demostrable sin importar las hipótesis que tengamos, de
%   manera que su regla de introducción es:
%  \begin{mathpar}
%   \inferrule*[]{
%   }{
%   \top\,\true
%   }
%  \end{mathpar}
% Dado que no tenemos información de c\'omo introducir la verdad, tampoco podemos
% tener información de como eliminarla, por lo que no hay regla de eliminación.


% \item Cuantificación Universal:
% ?`Bajo qu\'e circunstancias debe la fórmula~$\fa x\vp$ ser verdadera?,
% la respuesta depende claramente del dominio de cuantificación. Por
% ejemplo, si sabemos que la variable~$x$ toma como valores números naturales, 
% entonces podemos concluir que $\fa x\vp$ es verdadera si podemos probar que las 
% fórmulas $\vp[x:=0],\vp[x:=1],\ldots,\vp[x:=n],\ldots$.
% Siguiendo esta idea, debemos considerar que si todas las instancias de una 
% variable hacen que la f\'ormula sea verdadera, entonces una generalizaci\'on 
% tambi\'en lo es, lo cual nos lleva a la siguiente regla:
% \begin{mathpar}
%  \inferrule*[]{
%  \vp[x:=0]\;\true\and 
%  \vp[x:=1]\;\true \quad\ldots \and 
%  \vp[x:=n]\;\true \quad \ldots
%  }{
%  \fa x\vp\;\true
%  }
% \end{mathpar}

% Tal regla no es efectiva dado que tendríamos un número infinito de premisas, 
% usualmente se usa la regla de inducción en su lugar. Sin embargo la elección de 
% tal regla depende fuertemente de un dominio de cuantificación particular 
% mientras que lo que nos interesa es probar la verdad en cualquier dominio 
% posible. De manera que podremos decir que la f\'ormula~$\fa x\vp$ es verdadera 
% si al no asumir algo acerca de $x$, podemos cercioranos de la verdad de $\vp$, 
% es decir tenemos la siguiente regla informal:
% \begin{mathpar}
%  \inferrule*[]{
%  \vp\;\true\and x\;\text{ parámetrica en } \vp
%  }{
%  \fa x\vp\;\true
%  }
% \end{mathpar}
% Por otro lado si sabemos que la fórmula~$\fa x\vp$ es cierta entonces
% deberíamos poder concluir la verdad de $\vp[x:=t]$ para cualquier
% objeto~$t$, lo cual nos lleva a la siguiente regla:
% \begin{mathpar}
%  \inferrule*[]{
%  \fa x\vp\;\true
%  }{
%  \vp[x:=t]\;\true
%  }
% \end{mathpar}

% \item Cuantificación Existencial: Si sabemos que $\vp[x:=t]$ es cierta para 
% algún objeto~$t$ entonces podemos concluir que la f\'ormula~$\ex x\vp$ es 
% cierta, lo cual se modela mediante la regla:
% \begin{mathpar}
%  \inferrule*[]{
%  \vp[x:=t]\;\true
%  }{
%  \ex x\vp\;\true
%  }
% \end{mathpar}
% Por otro lado cuando conocemos la verdad de $\ex x\vp$ no sabemos cu\'al
% es el objeto~$t$ cuya existencia se asegura, de manera que sólo podemos
% asumir $\vp$ sin hacer ninguna suposición acerca del valor
% representado por $x$, lo cual se puede hacer sin problemas dado que la $x$ 
% estaba ligada en $A$. La regla de eliminación para el existencial es
% entonces similar al caso de la disyunción:
% \begin{mathpar}
%  \inferrule*[]{
% [\vp\;\true] \\\\
% \vdots \\\\
%  \ex x\vp\;\true \and \psi\;\true \and  x\notin FV(\psi)
% }{
% \psi\;\true
% }
% \end{mathpar}
% \ei

% Se observa que no hemos hablado de la negación, conectivo de suma
% importancia que discutiremos más tarde.
% Una convenci\'on es la equivalencia que considera una abreviatura de falso como
% $\vp\iff\psi  =_{def} (\vp\imp\psi) \land (\psi\imp\vp)$
% \\%por ahora
% %es conveniente tomar una convención para evitar ciertos
% %conectivos, la negación entre éllos: %los cuales se definirán a partir de 
% % otros:
% %\beqs
% %\ba{rll}
% %\vp\iff\psi & := & (\vp\imp\psi) \land (\psi\imp\vp) \\
% %\neg\vp     & := & \vp\imp\bot \\
% %\top        & := &  \neg\bot,\mbox{es decir, } \top:=\bot\imp\bot
% %\ea
% %\eeqs


% % \section{Carácter Constructivo de Conectivos y Cuantificadores}

% % Para poder poner en correspondencia el mundo de las pruebas formales con el
% % mundo de los programas computacionales debemos darle un caracter constructivo
% % a las primeras, de manera que puedan ser implementadas más adelante. Este
% % caracter constructivo se obtiene al olvidarnos de la noción clásica 
% % (platónica) de
% % verdad y en su lugar entender al juicio $\vp\,\true$ mediante la existencia
% % de una prueba constructiva de $\vp$, es decir la existencia de un método bien
% % definido que nos lleve a construir a $\vp$. Las siguientes reglas de
% % construcción se conocen como la \textbf{interpretación de 
% % Brouwer-Heyting-Kolmogorov (BHK)},
% % y su sabor algorítmico nos llevará más tarde a la famosa correspondencia de
% % Curry-Howard.
% % \bi
% % \item Una construcción de $\vp\land\psi$ consiste de una construcción de
% %   $\vp$ y una construcción de $\psi$.
% % \item Una construcción de $\vp\lor\psi$ consiste de una construcción de $\vp$
% %   ó de una construcción de $\psi$.
% % \item Una construcción de $\vp\imp\psi$ consiste de un método que transforma
% %   cada construcción de $\vp$ en una construcción de $\psi$.
% % \item Una construcción de $\fa x\vp$ es un método que transforma
% %   cualquier objeto $t$ en una construcción de $\vp[x:=t]$.
% %   \item Una construcción de $\ex x\vp$ es un par consistente de un
% %     objeto $t$ y una construcción de $\vp[x:=t]$.
% % \item No existe una construcción para $\bot$
% %   \item $\top$ siempre es construible.
% % \ei

%\section{Sistemas de Deducción Natural con Contextos}

En esta sección presentamos un sistema de deducción para L\'ogica de 
Predicados, con contextos o hipótesis localizadas, es decir, en cada paso de la 
deducción de una f\'ormula estarán disponibles todas las hipótesis 
representadas por un contexto. Las f\'ormulas ser\'an representadas por letras 
may\'usculas y los contextos por letras griegas may\'usculas.

Este sistema permite construir derivaciones de expresiones de la forma 
$\G\vdash A$ llamados \textbf{secuentes}, a diferencia del sistema de lógica 
ecuacional donde las expresiones derivadas son simplemente ecuaciones 
de fórmulas o de expresiones de un lenguaje particular. 
Esta presentación podría parecer más complicada que otras, sin embargo la 
disponibilidad de todo el conjunto de hipótesis en cada momento es de gran 
utilidad. % como se verá al estudiar sistemas de tipos.

% Las consideraciones hechas anteriormente se formalizan mediante un sistema de
% deducción natural para la lógica de predicados. Existen diversas formas de
% presentar estos sistemas, probablemente el lector ha visto con anterioridad el
% sistema mediante el método de cajas o de hipótesis etiquetadas. Nosotros
% usaremos una versión con hipótesis localizadas, es decir, en cada paso de la
% deducción estarán disponibles todas las hipótesis. Esta presentación podría
% parecer más complicada que otras, sin embargo la disponibilidad de todo el
% conjunto de hipótesis en cada momento será de gran utilidad al pasar al mundo
% de los programas. \\

\defin{Un contexto es una colección\footnote{Esta colección puede
    implementarse de distintas maneras, como lista, multiconjunto o
    conjunto. Nosotros consideramos que los contextos son listas.}
  finita de fórmulas $\vp_1,\ldots,\vp_n$.
 Usualmente denotaremos un contexto con $\G,\Delta,\Pi$. 
 En lugar de $\G\cup\Delta$ escribimos $\G;\D$. An\'alogamente $\G,\vp$ denota 
 al contexto $\G\cup\{\vp\}$. Es decir, la operación de unión de contextos se
 denota con punto y coma, mientras que la operación de agregar un elemento a
 un contexto se denota con coma.
}

En adelante hacemos la siguiente convención: siempre que un contexto sea de la 
forma $\G,A$, suponemos que la fórmula $A$ no figura o no aparece en $\G$.\\
Adem\'as, si $\G=\{\vp_1,\ldots,\vp_n\}$ entonces el conjunto de variables 
libres de~$\G$, denotado $FV(\G)$, se define como la unión de los conjuntos de
variables libres $FV(\vp_i)$ para cada $\vp_i \in \G$.

\section{Reglas de inferencia}
Recordemos que las reglas de inferencia permiten encadenar razonamientos, de 
esta forma se relacionan dos o varios secuentes mediante una l\'inea 
horizontal. Las reglas pueden leerse en dos sentidos: de arriba hacia abajo y 
viceversa, dependiendo del prop\'osito en el desarrollo de una demostraci\'on 
(generar, derivar o deducir informaci\'on). 

\medskip

Un secuente representa la relación de derivabilidad o deducibilidad 
$\mathbf{\Gamma \vdash A}$, leida como: 
\begin{center}
\enquote{la fórmula~$A$ es derivable o deducible en el contexto~$\G$}
\end{center}
Esta relaci\'on se define recursivamente a continuación mediante reglas que se 
clasifican en izquierdas y derechas, y que enfatizan cada conectivo que est\'a 
presente en el sistema.
\begin{description}
\item[Reglas derechas:] consideran cada forma sintáctica de la fórmula que
  está a la derecha del símbolo de derivabilidad $\vdash$. Estas reglas sirven
  para derivar fórmulas de manera directa de acuerdo a su conectivo
  principal, también se conocen como reglas de introducción. 
\item[Reglas izquierdas:] consideran cada forma sintáctica para una fórmula
  particular en el contexto, es decir, a la izquierda del símbolo de
  derivabilidad $\vdash$. 
\end{description}


Veamos cada regla particular:

\begin{itemize}
\item Regla\footnote{También llamado axioma.} inicial o de hipótesis:
\begin{mathpar}
  \inferrule*[right=(Hip)]{
  }{
  \scd{\G,A;\G'}{A}
  } 
\end{mathpar}
Es decir, una fórmula $A$ es derivable si en el contexto figura ella misma
como una de las hipótesis.

\item Reglas derechas:
 \begin{mathpar}
  \inferrule*[right=($\land\,$R)]{
  \scd{\G}{A} \and \scd{\G}{B}
  }{
  \scd{\G}{A\land B}
} 

\inferrule*[right=($\lor\,$R)]{
  \scd{\G}{A}
  }{
  \scd{\G}{A\lor B}
  }
  
  \inferrule*[right=($\lor\,$R)]{
  \scd{\G}{B}
  }{
   \scd{\G}{A\lor B}
   }\\\\
   
   \inferrule*[right=($\to\,$R)]{
  \scd{\G,A}{B}
  }{
  \scd{\G}{A\to B}
  }
 
  \inferrule*[right=($\exists\,$R)]{
  \scd{\G}{B[x:=t]}
  }{
  \scd{\G}{\ex x B}
}

  \inferrule*[right=($\fa\,$R)]{
  \scd{\G}{B\;\;\;\;\;x\notin FV(\G)}
  }{
  \scd{\G}{\fa x B}
}
\end{mathpar}

Obsérvese que estas reglas capturan el razonamiento matemático natural para
concluir una proposición de manera directa, de acuerdo a su operador 
principal. Tambien es importante observar que en el caso de la regla
($\land\,$R) que tiene dos premisas, los contextos de cada una de ellas deben
ser iguales. Esto se conoce como el estilo multiplicativo en reglas de
inferencia\footnote{En contraste con el estilo aditivo donde se permiten
  contextos distintos en las premisas y se construye el contexto unión en la conclusión.}. En ciertas
ocasiones, cuando los contextos son diferentes podemos transformarlos para que
sean iguales mediante el uso de las llamadas reglas estructurales que
enunciamos en la proposición~\ref{prop:reglasEstr}.

En el caso de la regla $(\exists R)$ el término~$t$ requerido en la premisa
debe ser propuesto por el usuario o agente que construye la prueba. En el caso
de $(\fa R)$ la condición lateral $x\notin FV(\G)$ corresponde al hecho de que
para concluir $\fa xB$ basta demostrar $B$ siempre y cuando el contexto no
\enquote{hable} de $x$, es decir, no contenga información particular acerca de 
$x$. 

\item Reglas izquierdas: 
\begin{mathpar}
\inferrule*[right=($\land$L)]{
  \scd{\G,A,B;\G'}{C}
  }{
  \scd{\G,A\land B;\G'}{C}
  }

    \inferrule*[right=($\lor\,$L)]{
  \scd{\G,A;\G'}{C} \and \scd{\G,B;\G'}{C}
  }{
  \scd{\G,A\lor B;\G'}{C}
  } \\\\
    
  \inferrule*[right=($\to$L)]{
  \scd{\G,A\to B;\G'}{A}
  }{
  \scd{\G,A\to B;\G'}{B}
  }

  \inferrule*[right=($\ex$L)]{
  \scd{\G,A;\G'}{C\;\;\;\;\;x\notin FV(\G,C;\G')}
  }{
  \scd{\G,\ex x A;\G'}{C}
  }

  \inferrule*[right=($\fa$L)]{
  \scd{\G,\fa x A, A[x:=t];\G'}{C}
  }{
  \scd{\G,\fa x A;\G'}{C}
  }
\end{mathpar}

Obsérvese que en
cada secuente hay dos contextos $\G$ y $\G'$ pero cualquier de ellos puede ser
vacío. Más aún, en el caso de la regla ($\fa\,$L) las dos fórmulas en el
contexto de la premisa se escriben juntas sin perder generalidad.

Las reglas izquierdas se enfocan en una fórmula particular del contexto para 
modificarlo y simplificar la construcción de una derivación. Por ejemplo, la
regla $(\lor L)$ corresponde al método de análisis de casos. En los casos para
la implicación y el cuantificador universal, las fórmulas correspondientes no desaparecen del contexto, como en los otros casos, pues
contienen información que puede ser usada para otros propósitos
posteriormente. El razonamiento capturado por la regla $(\exists L)$
corresponde al método usual de usar una hipótesis existencial, es decir de la
forma $\ex x A$, tratando al objeto $x$ que existe como una variable libre
cuya única propiedad conocida es que cumple $A$. Por eso se requiere la
condición de que ni el contexto $\G$ ni la conclusión $C$ \enquote{hablen} de 
$A$, es decir se debe cumplir $x\notin FV(\G,C)$. 
\end{itemize}


% \bi
% \item Implicación:
% \begin{mathpar}
% \inferrule*[right=($\to$ I)]{
% \Gamma,\vp\vdash \psi
% }{
% \Gamma\vdash \vp\to \psi
% }

% \inferrule*[right=($\to$ E)]{
% \Gamma\vdash \vp\to \psi\and 
% \Gamma\vdash \vp
% }{
% \Gamma\vdash \psi
% }
% \end{mathpar}

% \item Conjunción:
% \begin{mathpar}
% \inferrule*[right=($\land$ I)]{
% \G\vdash \vp\and \G\vdash\psi
% }{
% \G\vdash\vp\land \psi
% }

% \inferrule*[right=($\land$ E)]{
% \G\vdash\vp\land\psi
% }{
% \G\vdash\psi
% }

% \inferrule*[right=($\land$ E)]{
% \G\vdash\vp\land\psi
% }{
% \G\vdash\vp
% }
% \end{mathpar}

% \item Disyunción
% \begin{mathpar}
% \inferrule*[right=($\lor$ I)]{
% \G\vdash\vp
% }{
% \G\vdash\vp\lor\psi
% }

% \inferrule*[right=($\lor$ I)]{
% \G\vdash\psi
% }{
% \G\vdash\vp\lor\psi
% }

% \inferrule*[right=($\lor$ E)]{
% \G\vdash\vp\lor\psi \and \G,\vp\vdash \chi \and
% \G,\psi\vdash\chi
% }{
% \G\vdash\chi
% }
% \end{mathpar}

% \item Cuantificador Universal:
% \begin{mathpar}
% \inferrule*[right=($\fa$ I)]{
% \G\vdash \vp \and x\notin FV(\G)
% }{
% \G\vdash\fa x\vp
% }

% \inferrule*[right=($\fa$ E)]{
% \G\vdash \fa x\vp
% }{
% \G\vdash \vp[x:=t]
% }
% \end{mathpar}

% \item Cuantificador Existencial:
% \begin{mathpar}
% \inferrule*[right=($\ex$ I)]{
% \G\vdash \vp[x:=t]
% }{
% \G\vdash \ex x\vp
% }

% \inferrule*[right=($\ex$ E)]{
% \G\vdash\ex x\vp\and \G,\vp\vdash \psi\and x\notin FV(\G,\psi)
% }{
% \G\vdash\psi
% }
% \end{mathpar}

% \item Verdadero %(regla derivada):
% \beqs
% \inferrule*[]{}{\;\;\G\vdash \top\;\;}\;(\top I)
% \eeqs
%\ei

% Obsérvese que mediante estas reglas de inferencia no estamos derivando
% fórmulas sino expresiones de la forma $\G\vdash A$, conocidas como {\em 
% secuentes}.
% En particular las reglas de inferencia son correctas con respecto a la 
% consecuencia lógica, es decir transforman secuentes válidos
% (respecto a $\models$) en secuentes válidos como lo asegura la siguiente

% \begin{proposition}\label{prop:rdnsem}
% Sean $\G$ un contexto y $A,B,C$ fórmulas. 
% Se cumple lo siguiente
% \bi
% \item Si $\G,A\models B$ entonces $\G\models A\to B$.
% \item Si $\G\models A$ y $\G\models A\to B$ entonces $\G\models B$.
% \item $\G\models A\land B$ si y sólo si $\G\models A$ y $\G\models B$.
% \item Si $\G\models A$ entonces $\G\models A\lor B$.
% \item Si $\G\models B$ entonces $\G\models A\lor B$.
% \item Si $\G\models A\lor B,\;\G,A\models C$ y $\G,B\models C$ entonces 
% $\G\models C$.
% \item Si $\G\models A$ y $x\notin FV(\G)$ entonces $\G\models \fa xA$.
% \item Si $\G\models \fa xA$ entonces $\G\models A[x:=t]$ siendo $t$ cualquier 
% término.
% \item Si $\G\models A[x:=t]$ para algún término $t$ entonces 
% $\G\models\ex xA$.
% \item Si $\G\models\ex x A$ y $\G,A\models B$ donde $x\notin FV(\G,B)$ 
% entonces $\G\models B$.
% \ei
% \end{proposition}
% \proof Ejercicio \qed

\bigskip

La noción de prueba o derivación formal es similar a la usada en la lógica ecuacional:

\defin{Una derivación del secuente $\G\vdash A$ es una sucesión finita de 
secuentes $\G_1\vdash A_1,\ldots,\G_n\vdash A_n$ tal que:
\bi
\item $\G_i\vdash A_i$ es instancia de la regla $(Hip)$ ó
\item $\G_i\vdash A_i$ es conclusión de alguna regla de inferencia tal que 
las 
premisas necesarias figuran antes en la sucesión.
\item $\G\vdash A$ es el último elemento de la sucesión.
\ei
}

% \newpage
 
\noindent También es común ver a la derivaci\'on de un secuente como un
\'arbol, de acuerdo a la siguiente

\begin{definition}
  Una prueba formal de $\G\vdash A$ es un árbol finito, cuyos nodos están
  etiquetados por expresiones $\G'\vdash A'$ y satisfacen las siguientes
  condiciones:
  \begin{itemize}
  \item La etiqueta de la raíz es $\G\vdash A$.
  \item Todas las hojas están etiquetadas con instancias de la regla $(Hip)$.
  \item La etiqueta de un nodo padre se obtiene mediante la aplicación de una
  de las reglas de inferencia a los nodos hijos.
  \end{itemize}
\end{definition}

Las pruebas más relevantes son aquellas donde el contexto final está
vacío. Para esto introducimos la siguiente


\begin{definition}
  Si $\vdash A$ es derivable, es decir si $\varnothing\vdash A$ es
  derivable ($ A$ es derivable sin hipótesis) entonces decimos que $ A$ es 
  un teorema.
\end{definition}

Mostramos ahora algunas reglas estructurales que pueden ser de ayuda en la 
construcción de derivaciones:

\prop~\label{prop:reglasEstr}{Las siguientes reglas de inferencia son válidas:
  \bi
  \item Intercambio de premisas: 
  \begin{mathpar}
  \inferrule*[]{\G, A, B ;\G'\vdash  C}
    {\G, B, A;\G'\vdash  C} 
  \end{mathpar}

  \item Monotonía o debilitamiento: 
   \begin{mathpar}
    \inferrule*[]{\G;\G'\vdash A}
      {\G, B;\G'\vdash A}
   \end{mathpar}

  \item Contracción:
    \begin{mathpar}
    \inferrule*[]{\G, A, A;\G'\vdash B}
    {\G, A;\G'\vdash B}     
    \end{mathpar}

  \item Sustitución o Corte:
   \begin{mathpar}
    \inferrule*[]{\G, A;\G'\vdash B \and \G;\G'\vdash A}
    {\G;\G'\vdash B}    
   \end{mathpar}
  \ei
}

La regla de sustituci\'on o corte es de especial importancia ya que permite 
utilizar una f\'ormula o lema auxiliar~$ A$ en la demostraci\'on de la 
f\'ormula principal~$ B$. Al igual que el caso de la regla $(\ex R)$, la 
f\'ormula auxiliar~$ A$ debe ser propuesta por el usuario. 


\section{La Negación}

La negación es quizás el conectivo lógico más importante, recordemos
por ejemplo que para definir en lógica clásica todos los conectivos y
cuantificadores o para tener un conjunto completo de conectivos, basta 
quedarnos con uno de los conectivos binarios, un cuantificador y la negación, 
la cual es imprescindible. Sin embargo, el símbolo de negación $\neg$ puede
definirse dando distintas reglas de inferencias o axiomas y de acuerdo a los
mismos hablamos de distintas clases de negación. Sin importar que otros
conectivos estén presentes, discutimos hoara tres clases distintas de
negación: minimal, intuicionista o constructiva y clásica.

%\begin{itemize}
%\item Negación fuerte: $\dnp$ mas la regla $\lnot C$, lo cual recupera el
%  principio del tercero excluido. En tal caso decimos que la lógica es
%  clásica y denotamos al sistema con $\dnc$.
%\end{itemize}

%En particular en $\dnm$ el símbolo $\bot$ es una constante más sin
%propiedades particulares de manera que una fórmula con presencias de $\bot$
%puede o no ser demostrable en el sistema. En $\dnp$ la constante $\bot$ tiene
%una regla de eliminación de manera que algunas fórmulas que involucran a
%$\bot$ no demostrables en $\dnm$ lo serán en $\dnp$, pero no todas,  
%algunos ejemplos de fórmulas clásicas (es decir demostrables en
%$\dnc$) que NO son derivables en $\dnp$ son:
%\bi
%\item $\vp\lor\lnot\vp,\quad \; \lnot\lnot\vp\imp\vp$
%\item $\big((\vp\imp\psi)\imp\vp\big)\imp\vp$ 
%\item $(\lnot \psi\imp\lnot\vp)\imp (\vp\imp\psi)$
%\item $\lnot(\vp\land\psi)\imp\lnot\vp\lor\lnot\psi$
%\ei

\subsection{Lógica Minimal}

Se dice que la lógica es minimal si en el sistema de deducci\'on no hay 
reglas específicas para la negación~$\neg$ ni para la constante de falsedad~$\bot$. 
En un sistema minimal, la constante~$\bot$ está presente pero no tiene 
propiedades particulares. \\

En la presencia de $\bot$, el símbolo de negación se define como 
$$ \neg A =_{def} A\to\bot $$
% \noindent en cuyo caso hablamos de la negación constructiva, cuyas reglas de
% inferencia son:
En el caso de la negación no es sencillo definir un cálculo de secuentes con
reglas izquierdas y derechas. En su lugar definimos reglas de
introducción\footnote{Estas corresponden a las reglas 
derechas}, denotadas con {\sc I}, que permiten concluir una negación y reglas 
de eliminación, denotadas con {\sc E}, que permiten usar, de
manera indirecta, una negación para obtener más información. 
Las reglas de inferencia derivadas son:

\begin{mathpar}
\inferrule*[right=($\neg$ I)]{
\G,A\vdash\bot
}{
\G\vdash\neg A
}
% \G,A;\G'\vdash\bot
% }{
% \G;\G'\vdash\neg A
% }

\inferrule*[right=($\neg$ E$_m$)]{
\G\vdash A\and \G\vdash\neg A
}{
\G\vdash\bot
}
\end{mathpar}
Obsérvese que, de acuerdo a la definición de $\neg A$,  estas reglas
no son más que casos particulares de la regla derecha de implicación
\textsc{($\to$ R)} y la regla derivada correspondiente a 
\textit{modus ponens}, respectivamente.

%\item Ausencia de negación: $\dnp$ menos la regla $\bot E$, lo cual
%  nos lleva a eliminación de las reglas derivadas. $\lnot I,\lnot
%  E$. Si bien la regla $\lnot I$ es derivable, su contraparte $\lnot
%  E$ no lo es por lo que eliminamos ambas reglas para mantener la
%  simetría del sistema.\\
%  En tal caso decimos que la lógica es minimal y denotamos al sistema con 
%$\dnm$.

\subsection{Lógica Intuicionista}

La lógica intuicionista~\footnote{El nombre se debe a una corriente lógica
para fundamentar las matemáticas desarrollada a principios del siglo
XX.}  se obtiene al agregar a la lógica minimal la regla de eliminación de lo 
falso~$(\bot E)$ conocida también como \textit{ex-falso-quodlibet}.
\begin{mathpar}
\inferrule*[right=(EFQ)]{
\G\vdash\bot
}{\G\vdash A
}
\end{mathpar}

% La lógica intuicionista, al
% ser una extensión de la lógica minimal, respeta la interpretación BHK para
% los conectivos dados. Más aún, la lógica intuicionista modela la 
% intepretación BHK para la
% negación dada por: \textbf{ una construcción de $\lnot A$ consiste en un
%   método que convierte cada construcción de $ A$ en un objeto
%   inexistente},
% la cual puede modelarse definiendo a la negación como
% $\neg A:= A\imp\bot$.
% Obsérvese que esta interpretación es mucho más
% fuerte que pedir sólamente que no haya una construcción para $ A$. 
% La negación se rige por las siguientes reglas

% %\item Negación (reglas derivadas)
% \beqs
% \inferrule*[]{\G, A\vdash\bot}{\G\vdash\neg A}\;(\neg I)\quad \quad \;
% \inferrule*[]{\G\vdash\neg A\quad \G\vdash A}{\G\vdash B}\;(\neg E)
% \eeqs 

Se observa que cualquier fórmula derivada en la lógica minimal sigue
siendo derivable en la lógica intuicionista. Las reglas $(\neg I)$ y $(\neg
E_m)$ siguen siendo válidas. Además se pueden derivar
nuevas fórmulas, en particular una regla de eliminación de la negación:
% puede modificarse como sigue en la lógica intuicionista:
\begin{mathpar}
\inferrule*[right=($\neg$ E)]{
\G\vdash A\and \G\vdash \neg A
}{
\G\vdash B
}
\end{mathpar}
                                
Más aún, el carácter constructivo de la negación restringe a la lógica de una 
manera importante, en particular el sistema no permite probar la tautología 
clásica $  A\lor\lnot A $
conocida como el principio del tercero excluido. Para convencernos de tal
situación basta recordar qué significa el hecho de que una disyunción sea
demostrable. En el caso del tercero excluido tendríamos que construir 
una prueba de $ A$ o bien una prueba de $\neg A$ lo cual no es posible en
general. Este hecho implica igualmente que la fórmula~$\lnot\lnot A\imp A$
\textbf{NO} es válida. Por otro lado es fácil dar una derivación
de $ A\imp\lnot\lnot A$ desde la lógica minimal. \\
Otras fórmulas \textbf{NO} válidas en la lógica intuicionista son:
\bi
%\item $ A\lor\lnot A$.
\item $(\lnot A\imp\lnot B)\imp( B\imp A)$.
\item $( A\imp B)\lor( B\imp A)$.
\item $\lnot\fa x A\imp\ex x\neg A$.
\item $\fa x( A\lor B)\imp A\lor\fa x B$ con $x\notin FV( A)$.
\item $( B\imp\ex x A)\iff \ex x( B\imp A)$ con $x\notin FV( B)$.
\item $(\fa x A\imp B)\iff \ex x( A\imp B)$ con $x\notin FV( B)$.
\item $\fa x\lnot\lnot A\imp \lnot\lnot\fa x A$.
\ei

Las demostraciones de la invalidez intuicionista de tales fórmulas utilizan 
técnicas de semánticas de Heyting ó forzamiento mediante marcos que no 
pertenecen a nuestro curso.\\

Las lógicas minimal e intuicionista también se conocen como lógicas 
constructivas porque toda fórmula se puede construir o derivar directamente,
en particular se tienen las siguientes propiedades no válidas en la lógica 
clásica, donde $\vdash_i$ denota a la relación de derivabilidad en la lógica intuicionista:
\bi
 \item Propiedad Disyuntiva: Si $\vdash_{i} A\lor B$ entonces
  $\vdash_{i} A$ ó $\vdash_{i} B$.
 \item Propiedad Existencial: Si $\vdash_{i} \ex x A$ entonces existe un
  término $t$ tal que $\vdash_{i} A[x:=t]$.
\ei


\subsection{Lógica clásica}
Para recuperar a la lógica clásica tenemos que postular alguna de las
siguientes reglas:
\bi
\item Tercero Excluido~\footnote{También conocidacomo $(TND)$ por
  su nombre en latín \textit{Tertium non datur}.}
\begin{mathpar}
\inferrule*[right=(TE)]{
}{
\;\;\G\vdash A\lor\neg A \;\;
}
\end{mathpar}

\item Reducción al absurdo
\vspace*{-5pt}
\begin{mathpar}
\inferrule*[right=(RAA)]{
\G,\neg A\vdash\bot
}{
\G\vdash A}
% \G,\neg A;\G'\vdash\bot
% }{
% \G\vdash A}
\end{mathpar}
Esta última regla es muy utilizada en razonamientos matemáticos y se conoce
como reducción al absurdo. Cómparese con la regla $(\neg I)$ y reflexione por
qué son reglas distintas y en qué lógica son equivalentes.

\item Eliminación de la doble negación~\footnote{La regla dual para
    introducción de la doble negación es válida desde la lógica minimal:
\begin{mathpar}\inferrule*[]{\G\vdash A}{\G\vdash\neg\neg A}\;(\neg\neg I)\end{mathpar}}
\begin{mathpar}
\inferrule*[right=($\neg\neg$ E)]{
\G\vdash\neg\neg A
}{
\G\vdash A
}
\end{mathpar}
% \[
% %\inferrule*[]{A}{\neg\neg A}\;(\neg\neg I)\quad \quad \;\;
% \inferrule*[]{\neg\neg A}{A}\;(\neg\neg E)
% \]
\ei


% principio del tercero excluido, agregando la regla inicial del tercero 
% excluido
% \beqs
% \inferrule*[]{}{\G\vdash A\lor\lnot A}\;(TE)
% \eeqs
%  o equivalentemente agregando la ley de doble negación mediante la
%  siguiente regla:
% \beqs
% \inferrule*[]{\G\vdash\lnot\lnot A}{\G\vdash A}\;\;(\lnot C)
% \eeqs

Obsérvese que la regla \textsc{(TE)} permite probar $\vdash A\lor\lnot A$
situación imposible de motivar en el ámbito constructivo. Esta situación
rompe con la simetría de los conectivos dada por las reglas izquierdas y
derechas. En particular en la lógica clásica podemos deducir disyunciones
por medio de una regla distinta a la regla derecha para la
disyunción, a saber mediante el uso de la regla del tercero excluido. 
% En lugar de la
% regla anterior también podemos usar la regla de prueba por
% contradicción o reducción al absurdo:
% \beqs
% \inferrule*[]{\G,\lnot\vp \vdash\bot}{\G\vdash\vp}\;\;(\neg RAA)
% \eeqs
%\\ La
%interpretación de la negación en el mundo de los programas es un punto muy
%discutido todavía en el ámbito de la investigación. %Al nivel de este curso
%podemos decir con ciertos riesgos que la correpondencia entre el múndo de las
%pruebas y el de los programas se rompe al agregar la negación.
% En este sistema podemos seguir viendo a la negación $\neg A$ como una 
% abreviatura de $A\to\bot$ o bien considerar que la negación es primitiva en 
% cuyo caso la regla de eliminación de la negación no es un caso particular del 
% modus ponens y debe agregarse al sistema:
% \[
% \inferrule*[]{A\quad \;\;\neg A}{B}\;(\neg E)
% \]
% A continuación damos las reglas correspondientes a la falsedad y a las
% negaciones constructiva y clásica.

%\bi
%  \item Lógica minimal: negación constructiva $\neg A=_{def} A\to\bot$.
% \beqs
% \inferrule*[]{\G,A\vdash\bot}{\G\vdash\neg A}\;(\neg I)\quad \quad \quad 
% \inferrule*[]{\G\vdash A\quad \;\;\G\vdash\neg A}{\G\vdash\bot}\;(\neg E)
% \eeqs
% \item Lógica intuicionista: ex-falso-quodlibet
% \beqs
% \inferrule*[]{\G\vdash\bot}{\G\vdash\vp}\;(\bot E)
% \eeqs
% \item Lógica clásica: alguna de las siguientes
% \bi
% \item Tercero excluido:
% \beqs
% \inferrule*[]{}{\G\vdash A\lor\neg A}\;(TE)
% \eeqs
% \item Reducción al absurdo:
% \beqs
% \inferrule*[]{\G,\neg A\vdash\bot}{\G\vdash A}\;(RAA)
% \eeqs
% \item Eliminación de la doble negación:
% \beqs
% \inferrule*[]{\G\vdash\neg\neg A}{\G\vdash A}\;(\neg\neg E)
% \eeqs
% \ei
% \ei

% también debe agregarse la regla de eliminación de la negación:

% \[
% \inferrule*[]{\G\vdash A\quad \;\;\G\vdash\neg A}{\G\vdash B}\;(\neg E)
% \]


\subsection{Otras reglas de la negación clásica}

Las siguientes reglas son de utilidad en la lógica clásica. Se deja como 
ejercicio mostrar que son derivables a partir de las reglas de negación dadas.

\begin{itemize}
\item Modus Tollens:
\begin{mathpar}
\inferrule*[right=MT]{
\G\vdash A\to B\and \G\vdash \neg B
}{
\G\vdash \neg A}
\end{mathpar}
\item Silogismo disyuntivo:
\begin{mathpar}
\inferrule*[right=SD]{
\G\vdash A\lor B\and 
\G\vdash \neg A
}{
\G\vdash B
}
\end{mathpar}
\end{itemize}


\section{Ejemplos de derivaciones}

En lo que sigue denotamos con $\vdash_m,\vdash_i,\vdash_c$ a las
relaciones de derivación en los sistemas minimal, intuicionista y clásico, 
respectivamente.
De las definiciones, es claro que el sistema intuicionista es una
extensión conservativa del minimal y el clásico del intuicionista. Es
decir, $\G\vdash_m A$ implica $\G\vdash_i A$ implica $\G\vdash_c A$. Sin
embargo ninguna de las afirmaciones recíprocas es válida en
general. En los ejemplos siguientes debe entenderse que el sistema
correspondiente es estrictamente necesario, es decir,  para las
derivaciones en $\vdash_i$ (respectivamente $\vdash_c$) no existe una 
derivación en $\vdash_m$ (respectivamente $\vdash_i$), aunque para mostrar 
formalmente estas afirmaciones se necesitan técnicas semánticas que van más
allá del alcance de nuestro curso.

Las derivaciones siguientes se presentan de forma lineal, a diferencia de los 
\'arboles de derivaci\'on introducidos al inicio de la nota.
En una derivaci\'on lineal, el \'ultimo paso es el secuente a demostrarse
y cada paso es justificado al ser la conclusic\'on de la regla aplicada 
junto con los pasos usados como premisas.
% \prop{Se cumplen las siguientes derivaciones:
% \bi
% \item $\vdash_m A\imp \neg\neg A$
% \item $\vdash_m \neg\neg(A\lor\neg A)$
% \item $\vdash_m (A\imp B)\imp (A\imp\neg B)\imp \neg A$
% \item $\vdash_m \neg(A\land\neg A)$
% \item $\vdash_m (\neg A\imp A)\imp \neg\neg A$
% \item $\vdash_m A\land\neg B\imp \neg(A\imp B)$
% \item $\vdash_m (A\imp\neg A)\imp\neg A$
% \item $\vdash_m (A\imp B)\imp (\neg B\imp \neg A)$
% \item $\vdash_m (A\imp\neg B)\iff (B\imp\neg A)$
% \item $\vdash_m\neg(A\lor B)\iff \neg A\land\neg B$
% \item $\vdash_m \neg A\lor\neg B\imp \neg(A\land B)$
% \item $\vdash A\imp \neg A\imp B$
% \item $\vdash A\lor B\imp \neg A\imp B$
% \item $\vdash \neg A\lor B\imp A\imp B$
% \item $\vdash A\lor\neg A\imp \neg\neg A\imp A$
% \item $\vdash \neg\neg A\imp \neg A\imp A$
% \item $\vdash A\imp (B\iff \neg A\lor B)$
% \item $\vdash_c \neg\neg A\iff A$
% \item $\vdash_c (A\imp B)\iff \neg B\imp\neg A$
% \item $\vdash_c \neg(A\land B)\iff \neg A\lor\neg B$
% \item $\vdash_c (A\imp B)\iff \neg A\lor B$
% \item $\vdash_c (A\imp B)\lor (B\imp A)$
% \item $\vdash_c ((A\imp B)\imp A)\imp A$
% \ei
% }
% \proof Probamos sólo algunos incisos, dejando los restantes como ejercicio.

\bi
\item Mostrar que: \hspace{0.5cm} $\vdash_m (p \land q \to r) 
\to p \to q \to r$ 
% Basta mostrar:  $ (p \land q \to r) \; , \;  p \; , \; q \vdash r$ 
\[
\begin{array}{rll}
1\quad & p \land q \to r \; , \; p \; , \; q \vdash p & 
(Hip) \\
2\quad & p \land q \to r \; , \; p \; , \; q \vdash q & 
(Hip)\\
3\quad & p \land q \to r \; , \; p \; , \; q \vdash p \land q & 
( \land R) \;\; 1,2\\
4\quad & p \land q \to r \; , \; p \; , \; q \vdash  r 
& (\to L) \;\; 3\\
5\quad & p \land q \to r \; , \; p  \vdash q \to r 
& (\to R) \;\; 5 \\
6\quad & p \land q \to r  \vdash p \to q \to r &
(\to R) \;\; 6\\
7\quad &  \vdash (p \land q \to r) \to p \to q 
\to r &(\to R) \;\; 7
\end{array}
\]

\item Sea $\G=\{p \to q \lor r \; , \; q\to r \; , \; r\to s\}$, queremos 
mostrar $\G\vdash_m p\to s$ 
\[
\begin{array}{rll}
1 \quad & \G,\;p,\;q \vdash \; q & \qquad (Hip) \\
2 \quad & \G,\;p,\;q\vdash\; r & \qquad (\to L) \; 1 \\
3 \quad & \G,\;p,\;q\vdash\; s & \qquad ( \to L) \; 2 \\
4 \quad & \G,\;p,\;r \vdash\; r & \qquad (Hip)    \\
5 \quad & \G,\;p,\;r \vdash \; s & \qquad (\to L) \; 4 \\
6 \quad & \G,\;p,\; q\lor r \; \vdash \; s & \qquad  (\lor L) \;3,5\\
7 \quad & \G,\;p\; \vdash \; p & \qquad  (Hip)  \\
8 \quad & \G,\;p\; \vdash \; q\lor r & \qquad  (\to L) \; 6 \\
9 \quad & \G,\;p \;\vdash \; s & \qquad  (cut)\; 5,7 \\
10\quad & \G\;\vdash \; p\to s &\qquad (\to R) \; 8
\end{array}
\]



\item Demostrar que $\vdash_m A\imp \neg\neg A$, aplicando la definición de 
negación y la regla derecha de la implicaci\'on basta mostrar que  
$A,A\imp\bot\vdash\bot$:
\[
\ba{rll}
1 & A, A\imp\bot\vdash A & (Hip)\\
2 & A, A\imp\bot\vdash \bot & (\imp L) \; 1\\
\ea
\]

\item Demostrar que $\vdash_m\neg\neg(A\lor\neg A)$, aplicando la definición de 
negación y la regla derecha de la implicaci\'on basta derivar
$A\lor\neg A\imp \bot\vdash_m\bot$:
\[
\ba{rll}
1. & A\lor \neg A\imp\bot,A\vdash A & (Hip)\\
2. & A\lor \neg A\imp\bot,A\vdash A\lor\neg A & (\lor R)\;1\\
3. & A\lor \neg A\imp\bot,A\vdash \bot & (\imp L)\;2\\
4. & A\lor \neg A\imp\bot\vdash A\imp \bot & (\imp R)\;3 \quad \lnot 
A=_{def}A\to \bot\\
5. & A\lor \neg A\imp\bot\vdash A\lor \neg A & (\lor R)\;4\\
6. & A\lor \neg A\imp\bot\vdash \bot & (\imp L)\; 5\\
\ea
\]

\begin{comment}

\item Demostrar el siguiente teorema 
$\vdash_m \neg(A\lor B)\iff \neg A\land\neg B$. Hay que mostrar ambas 
implicaciones:
\be
\item $\vdash_m \neg(A\lor B)\imp \neg A\land\neg B$. Basta mostrar
$A\lor B\imp \bot,A\vdash_m\bot$ y $A\lor B\imp \bot,B\vdash_m\bot$.
\[
\ba{rll}
1. & A\lor B\imp\bot,A\vdash A & (Hip)\\
2. & A\lor B\imp\bot,A\vdash A\lor B & (\lor I)\;1\\
3. & A\lor B\imp\bot,A\vdash A\lor B\imp\bot & (Hip)\\
4. & A\lor B\imp\bot,A\vdash \bot & (\imp E)\;2,3\\
\ea
\]
La derivación faltante es análoga.




\item Para demostrar que $\vdash_m \neg A\land\neg B\imp \neg(A\lor B)$, basta 
mostrar $\neg A\land\neg B,A\lor B\vdash \bot$.
\[
\ba{rll}
1. & \neg A\land \neg B, A\lor B\vdash A\lor B & (Hip)\\
2. & \neg A\land \neg B, A\lor B,A\vdash \neg A\land\neg B & (Hip)\\
3. & \neg A\land \neg B, A\lor B,A\vdash A & (Hip)\\
4. & \neg A\land \neg B, A\lor B,A\vdash \neg A & (\land E)\;2\\
5. & \neg A\land \neg B, A\lor B,A\vdash \bot & (\imp E)\;3,4\\
6. & \neg A\land \neg B, A\lor B,B\vdash \neg A\land\neg B & (Hip)\\
7. & \neg A\land \neg B, A\lor B,B\vdash B & (Hip)\\
8. & \neg A\land \neg B, A\lor B,B\vdash \neg B & (\land E)\;6\\
9. & \neg A\land \neg B, A\lor B,B\vdash \bot & (\imp E)\;7,8\\
10. & \neg A\land \neg B, A\lor B\vdash \bot & (\lor E)\;1,5,9\\
\ea
\]
\ee

\end{comment}

\item Demostrar el teorema $\vdash_i \neg A\lor B\imp A\imp B$.
\[
\ba{rll}
1 & \neg A,\; A \vdash A & (Hip)\\
2 & \neg A,\; A \vdash \neg A & (Hip)\\
3 & \neg A, \;A \vdash B & (\bot E)\; 1,2\\
4 & B,\; A\vdash B & (Hip)\\
5 & \neg A\lor B,\; A\vdash B & (\lor L)\; 3,4\\
6 & \neg A\lor B \vdash A \to B & (\to R) \;6  \\
7 &      \vdash \neg A\lor B \to A \to B & (\to R) \; 6
\ea
\]

\begin{comment}

\item Para demostrar $\vdash_i A\lor\neg A\imp \neg\neg A\imp A$ basta mostrar
$A\lor\neg A,\neg\neg A\vdash A$.
\[
\ba{rll}
1. & A\lor \neg A,\neg \neg A\vdash A\lor \neg A & (Hip)\\
2. & A\lor \neg A,\neg \neg A,A\vdash A & (Hip)\\
3. & A\lor \neg A,\neg \neg A,\neg A\vdash \neg A & (Hip)\\
4. & A\lor \neg A,\neg \neg A,\neg A\vdash \neg\neg A & (Hip)\\
5. & A\lor \neg A,\neg \neg A,\neg A\vdash \bot & (\imp E)\;3,4\\
6. & A\lor \neg A,\neg \neg A,\neg A\vdash A & (\bot E)\;5\\
7. & A\lor \neg A,\neg \neg A\vdash A & (\lor E)\;1,2,6\\
\ea
\]

\item El teorema $\vdash_c \neg\neg A\iff A$ se demuestra en dos partes. La 
parte $\vdash A\imp\neg\neg A$ ya fue probada. Basta probar 
entonces $\vdash\neg\neg A\imp A$, es decir, $\neg\neg A\vdash
A$. Pero esto es inmediato por la regla $(\neg\neg E)$.\\

\end{comment}

\item Para el teorema $\vdash_c \neg(A\land B)\iff \neg A\lor\neg B$ se tienen 
dos partes: la parte \enquote{$\leftarrow$} es válida minimalmente y por lo 
tanto v\'alida en l\'ogica cl\'asica (se deja como ejercicio); para la otra 
parte hay que derivar 
$\neg(A\land B)\vdash_c \neg A\lor \neg B$ :

\[
\ba{rll}
1 & \neg(A\land B), A,B\vdash A & (Hip)\\
2 & \neg(A\land B), A,B\vdash B & (Hip)\\
3 & \neg(A\land B), A,B\vdash A\land B & (\land R)\;1,2\\
4 & \neg(A\land B), A,B\vdash \neg(A\land B) & (Hip)\\
5 & \neg(A\land B), A,B\vdash \neg A\lor\neg B & (\lnot E )3,4\;\\
6 & \neg(A\land B), A,\neg B\vdash \neg B & (Hip)\\
7 & \neg(A\land B), A,\neg B\vdash \neg A\lor\neg B & (\lor R)\; 6 \\
8 & \neg(A\land B), A,B\lor\neg B\vdash \neg A\lor\neg B & (\lor L)\; 5,7 \\
9 & \neg(A\land B), A\vdash B\lor\neg B & (TE)\\
10 & \neg(A\land B), A\vdash \neg A\lor\neg B & (cut)\; 8,9 \\
11 & \neg(A\land B), \neg A\vdash \neg A & (Hip)\\
12 & \neg(A\land B), \neg A \vdash \neg A\lor\neg B & (\lor R)\; 11 \\
13 & \neg(A\land B), A\lor \neg A \vdash \neg A\lor\neg B & (\lor L)\; 10,12\\
14 & \neg(A\land B)\vdash A\lor \neg A & (TE)\\
15 & \neg(A\land B)\vdash \neg A\lor\neg B & (cut)\; 14,13\\
\ea
\]

\item $\vdash_c ((A\imp B)\imp A)\imp A$. Por la ley de 
contrapositiva~\footnote{Se puede mostrar que esta ley es un teorema, es decir 
que $\vdash (A\imp B) \iff (\lnot B\imp \lnot A)$.} basta 
mostrar  $\neg A\vdash_c \neg((A\imp B)\imp A)$. Por otra parte, se puede
probar que $C\land\neg D\vdash_m \neg (C\imp D)$ para cualesquiera f\'ormulas 
$C$ y $D$.
Por lo que basta mostrar $\neg A\vdash_c (A\imp B)\land\neg A$, lo 
cual se sigue de $\neg A, A\vdash_c B$ y que es inmediato de la 
regla~$(\bot E)$.

\ei


Veamos ahora algunos ejemplos con cuantificadores:

\bi
\item Mostrar que:
\[
\vdash_m \forall w (Pw \to Qw) 
\to \forall x (\exists y (P y \land Rxy) \to \exists z (Qz 
\land Rxz))
\]
\[
\begin{array}{rll}
1 \quad & \forall w (Pw \to Qw) \, , \, \exists y (P y \land Rxy) \, , 
\, P y,\, Rxy,\, P y \to Qy  \vdash Rxy  & (Hip)\\
2 \quad & \forall w (Pw \to Qw) \, , \, \exists y (P y \land Rxy) \, , 
\, P y ,\, Rxy,\, P y \to Qy  \vdash Py  & (Hip) \\
3 \quad & \forall w (Pw \to Qw) \, , \, \exists y (P y \land Rxy) \, , 
\, P y ,\, Rxy,\, P y \to Qy  \vdash Q y  & (\to L) \; 2\\
4 \quad & \forall w (Pw \to Qw) \, , \, \exists y (P y \land Rxy) \, , 
\, P y ,\, Rxy,\, P y \to Qy  \vdash Q y \land Rxy & (\land R) \; 1,3\\
5 \quad & \forall w (Pw \to Qw) \, , \, \exists y (P y \land Rxy) \, , 
\, P y \land Rxy ,\, P y \to Qy \vdash Qy \land Rxy  & (\land L) \; 4 \\
6 \quad &  \forall w (Pw \to Qw) \, , \, \exists y (P y \land Rxy)\, , 
\, P y \land Rxy,\, P y \to Qy  \vdash \exists z ( Qz \land Rxz) & (\exists R) 
\; 5\\
& \quad \quad z\notin FV(\{\forall w (Pw \to Qw) \, , \, \exists y (P y \land 
Rxy)\, , \, P y \land Rxy,\, P y \to Qy \}) & \\
7 \quad &  \forall w (Pw \to Qw) \, , \, \exists y (P y \land Rxy)\, , 
\, P y \land Rxy \vdash \exists z ( Qz \land Rxz) & (\forall L) \; 6 [w:=y]\\
8\quad &  \forall w (Pw \to Qw) \, , \, \exists y (P y \land Rxy) \vdash 
\exists z ( Qz \land Rxz)  & (\exists L) \; 7 \\
& \quad \quad y\notin FV(\{\forall w (Pw \to Qw) \, , \, \exists y (P y \land 
Rxy),\exists z ( Qz \land Rxz)\}) &\\
9\quad &  \forall w (Pw \to Qw)  \vdash 
\exists y (P y \land Rxy) \to \exists z ( Qz \land Rxz)  & (\to R \; ) 8 \\
10\quad &  \forall w (Pw \to Qw)  \vdash 
\forall x (\exists y (P y \land Rxy) \to \exists z ( Qz \land Rxz)) 
    & (\forall R)\; 9 \\
& \quad \quad x\notin FV(\{\forall w (Pw \to Qw)\})\\
11\quad & \vdash 
\forall w (Pw \to Qw) \to \forall x (\exists y (P y \land Rxy) 
\to \exists z ( Qz \land Rxz)) & (\to R) \; 10\\
\end{array}
\]


% \item $\vdash_m  A\lor\fa x B\imp\fa x( A\lor B)$ con $x\notin FV( A)$.
% \[
% \ba{rll}
% 1. &  A\lor\fa x B, A\vdash  A\lor\fa x B & (Hip)\\\vspace*{5pt}
% 2. &  A\lor\fa x B, A\vdash  A & (Hip)\\\vspace*{5pt}
% 3. &  A\lor\fa x B, A\vdash  A\lor B & (\lor I,2)\\\vspace*{5pt}
% 4. &  A\lor\fa x B, A\vdash \fa x( A\lor B) 
%   & (\fa I,3),\;x\notin FV(\{ A\lor\fa x B, A\})\\\vspace*{5pt}
% 5. &  A\lor\fa x B,\fa x B\vdash \fa x B & (Hip)\\\vspace*{5pt}
% 6. &  A\lor\fa x B,\fa x B\vdash  B & (\fa E,5)\\\vspace*{5pt}
% 7. &  A\lor\fa x B,\fa x B\vdash  A\lor B & (\lor I,6)\\\vspace*{5pt}
% 8. &  A\lor\fa x B,\fa x B\vdash \fa x( A\lor B) 
%   & (\fa I,7),\;x\notin FV(\{ A\lor\fa x B,\fa x B\})\\\vspace*{5pt}
% 9. &  A\lor\fa x B\vdash \fa x( A\lor B) & (\lor E,1,4,8)
% \ea
% \]
% 
% \item Para demostrar que 
% $\vdash_c\fa x( A\lor B)\imp A\lor\fa x B$ con $x\notin FV( A)$, basta derivar 
% $\fa x( A\lor B),\neg A\vdash_c\fa x B$
% \[
% \ba{rll}
% 1. & \fa x( A\lor B),\neg A\vdash \fa x( A\lor B) & (Hip)\\
% 2. & \fa x( A\lor B),\neg A\vdash \neg A & (Hip)\\
% 3. & \fa x( A\lor B),\neg A\vdash  A\lor B & (\fa E,1)\\
% 4. & \fa x( A\lor B),\neg A\vdash  B & (RB,2,3)\text{\footnotemark}\\
% 5. & \fa x( A\lor B),\neg A\vdash \fa x B & (\fa I,4),x\notin FV(\{\fa 
% x( A\lor B),\neg A\})
% \ea
% \]
% \footnotetext{Aquí $RB$ denota a la regla de resolución binaria, válida en 
% lógica clásica, se deja como ejercicio su demostraci\'on.}
% %\item $( B\imp\ex x A)\iff \ex x( B\imp A)$ con $x\notin FV( B)$.


\item Demostrar que para cualesquiera f\'ormulas $A$ y $B$ 
con $x\notin FV( B)$, se cumple que 
  $\vdash_m \ex x( A\imp B)\imp (\fa x A\imp B)$
%   Basta ver que $\ex x( A\imp B),\;\fa x A\vdash_m  B$

\[
\ba{rll}
1. &  A\imp B,\;\fa x A,  A\vdash  A & (Hip)\\
2. &  A\imp B,\;\fa x A,  A\vdash  B & (\to L) 1\\
3. &  A\imp B,\;\fa x A \vdash  B & (\fa L) 2\\
4. & \ex x( A\imp B),\;\fa x A\vdash   B 
 & (\ex E) 3 \;x\notin FV(\{\ex x( A\imp B), B\})\\
5. & \ex x( A\imp B)\vdash \fa x A \to  B  & (\to R) 4\\
6. & \vdash \ex x( A\imp B) \to \fa x A \to   B  & (\to R) 5 \\
\ea
\]

\begin{comment}
% \item $\vdash_c(\fa x A\imp B)\iff \ex x( A\imp B)$ con $x\notin
%   FV( B)$.
% \bi
\item $\vdash_c (\fa x A\imp B)\imp \ex x( A\imp B)$ con $x\notin
  FV( B)$.  Puesto que estamos en lógica clásica basta probar
$\fa x A\imp B\vdash_c \neg\fa x\neg( A\imp B)$, es decir, 
$\fa x A\imp B,\;\fa x\neg( A\imp B)\vdash_c \bot$.

\[
\ba{rll}
1. & \fa x A\imp B,\;\fa x\neg( A\imp B)\vdash \fa
x\neg( A\imp B)  & (Hip)\\
2. & \fa x A\imp B,\;\fa x\neg( A\imp B)\vdash \neg( A\imp B)
& (\fa E,1)\\
3. & \fa x A\imp B,\;\fa x\neg( A\imp B)\vdash  A\land\neg  B & 
(equiv.\;
logica, 2) \\
4. & \fa x A\imp B,\;\fa x\neg( A\imp B)\vdash  A  & (\land E,3)
\\
5. & \fa x A\imp B,\;\fa x\neg( A\imp B)\vdash \fa x A  & (\fa
I,4),\;x\notin FV(\{\fa x A\imp B,\;\fa x\neg( A\imp B)\}) \\
6. & \fa x A\imp B,\;\fa x\neg( A\imp B)\vdash \fa x A\imp B
& (Hip)\\
7. & \fa x A\imp B,\;\fa x\neg( A\imp B)\vdash  B  & (\imp
E,5,6)\\
8. & \fa x A\imp B,\;\fa x\neg( A\imp B)\vdash\neg  B & (\land E,
3) \\
9. & \fa x A\imp B,\;\fa x\neg( A\imp B)\vdash\bot & (\imp E, 7,8) \\
\ea
\]
%\item $\fa x\lnot\lnot A\imp \lnot\lnot\fa x A$.

\item $\vdash_m\fa xA\imp\neg\ex x\neg A$. 
Basta ver que $\fa xA,\;\ex x\neg A\vdash_m\bot$
\[
\ba{rll}
1. & \fa x A,\ex x\neg A\vdash \ex x\neg A & (Hip).\\
2. & \fa x A,\ex x\neg A,\neg A\vdash \neg A & (Hip).\\
3. & \fa x A,\ex x\neg A,\neg A\vdash \fa x A & (Hip).\\
4. & \fa x A,\ex x\neg A,\neg A\vdash A & (\fa E,3).\\
5. & \fa x A,\ex x\neg A,\neg A\vdash \bot & (\imp E,2,4).\\
6. & \fa x A,\ex x\neg A\vdash \bot & (\ex E,1,5),\;x\notin FV(\{\fa x A,\ex 
x\neg A,\bot\}).\\
\ea
\]
\end{comment}

\item $\vdash_c \neg\ex x\neg A\imp \fa x A$. 
Basta ver que $\neg\ex x\neg A\vdash_c\fa xA$ y como 
$x\notin FV(\neg\ex x\neg A)$ basta con $\neg\ex x\neg A\vdash_c A$, para lo 
cual mostramos $\neg\ex x\neg A\vdash_c \neg\neg A$, es decir 
$\neg\ex x\neg A,\neg A\vdash_c \bot$
\[
\ba{rll}
1 & \neg\ex x\neg A,\neg A\vdash \neg A & (Hip)\\
2 & \neg\ex x\neg A,\neg A\vdash \ex x\neg A & (\ex R) 1 \quad [x:=x]\\
3 & \neg\ex x\neg A,\neg A\vdash \bot & (\imp L) 2 \quad \neg\ex x\neg A 
=_{def} \ex x\neg A \to \bot \\
\ea
\]


\ei


\section{Estrategias de derivación}
Las siguientes estrategias se basan en las reglas derechas y permiten 
construir una fórmula de acuerdo a su conectivo principal. 

\begin{itemize}
\item Para derivar $\G\vdash A\to B$ basta derivar $\G,A\vdash B$.
\vspace*{5pt}
\item Para derivar  $\G\vdash A\land B$ basta derivar $\G\vdash A$ y 
$\G\vdash B$
\vspace*{5pt}
\item Para derivar $\G\vdash A\lor B$ basta derivar $\G\vdash A$ o bien 
$\G\vdash B$
\vspace*{5pt}
\item Para derivar $\G\vdash \forall x A$ basta derivar $\G\vdash A$ donde 
s.p.g. $x\notin FV(\G)$
\vspace*{5pt}
\item Para derivar $\G\vdash \exists x A$ basta encontrar un término $t$ tal 
que $\G\vdash A[x:=t]$
\end{itemize}


Las siguientes estrategias se basan en las reglas izquierdas y permiten 
construir una fórmula $C$ usando una premisa particular:
\bi
\item Para derivar $\G,A\to C;\G'\vdash C$, basta derivar 
$ \G,A\to C;\G'\vdash A $

\item Para derivar $\G,A\land B;\G'\vdash C$ basta derivar 
$ \G,A,B;\G'\vdash C $

\item Para derivar $\G,A\lor B;\G'\vdash C$ basta derivar
$\G,A;\G'\vdash C\text{ y }\G,B;\G'\vdash C $

% \item Para derivar $\G,A\to B\vdash B$ basta derivar  $\G,A\to B\vdash A$
% \vspace*{5pt}
%\item Para derivar $\G,\exists x A\vdash C$ basta derivar $\G,A[x:=t]\vdash C$
\item Para derivar $\G,\ex x A;\G'\vdash C$ basta derivar 
$ \G,A;\G'\vdash C$ donde s.p.g. $x\notin FV(\G;\G',C)$
\item Para derivar $\G,\fa x A;\G'\vdash C$ basta proponer un término $t$ y derivar 
$\G,\fa x A, A[x:=t];\G'\vdash C$.

\ei
% \item Corte: Para derivar $\G\vdash C$ basta proponer $A$ y derivar
% \be
% \item $\G\vdash A\to C$
% \vspace{0.1cm}
% \item $\G\vdash A$
% \vspace*{5pt}
% \ee


% Las siguientes estrategias se basan en las reglas de eliminación y permiten 
% construir una fórmula $C$ usando una premisa particular:
% \bi
% \item Aplicación: para derivar $\G,A\to C\vdash C$, basta derivar 
% $ \G,A\to C\vdash A $

% \item Para derivar $\G,A\land B\vdash C$ basta derivar 
% $ \G,A,B\vdash C $

% \item Para derivar $\G,A\lor B\vdash C$ basta derivar
% $\G,A\vdash C\text{ y }\G,B\vdash C $

% % \item Para derivar $\G,A\to B\vdash B$ basta derivar  $\G,A\to B\vdash A$
% % \vspace*{5pt}
% %\item Para derivar $\G,\exists x A\vdash C$ basta derivar $\G,A[x:=t]\vdash C$
% \item Para derivar $\G,\ex x A\vdash C$ basta derivar 
% $ \G,A\vdash C$ donde $x\notin FV(\G,C)$
% \ei
% % \item Corte: Para derivar $\G\vdash C$ basta proponer $A$ y derivar
% % \be
% % \item $\G\vdash A\to C$
% % \vspace{0.1cm}
% % \item $\G\vdash A$
% % \vspace*{5pt}
% % \ee

La siguiente estrategia corresponde al uso de un lema, la fórmula~$A$, como 
se usa en matemáticas al demostrar un teorema usando resultados previos. 
Se recomienda utilizarla cuando las anteriores no funcionan directamente.
\bi
\item Aserción: Para derivar $\G\vdash C$ basta proponer $A$ y derivar tanto
$\G\vdash A$  como  $\G,A\vdash C$ .
\ei

El uso adecuado de las estrategias anteriores nos llevar eventualmente  a 
buscar pruebas más sencillas que las originales. Las siguientes estrategias 
permiten reducir el número de pruebas buscadas y terminar el proceso de búsqueda.
% permiten concluir 
% pruebas o disminuir el número de subpruebas en una prueba particular.
\bi
\item Para derivar $ \G,A;\G'\vdash A $
 no hay nada más que hacer pues esta es una derivación válida.
\item Para derivar $\G,\forall xA;\G'\vdash A[x:=t]$ no hay nada más que 
hacer pues esta es una derivación válida.
%\vspace{2cm}
% \item Si $\G,A\vdash C$ entonces $\G,\exists x A\vdash C$ donde $x\notin 
% FV(\G,C)$
% \espc
% \item 
\ei

\section{Tácticas}
Las estrategias anteriores pueden mecanizarse mediante un procedimiento de 
búsqueda de pruebas orientado a metas. Una meta es simplemente un 
secuente~$\G\vdash A$ correspondiente a la prueba deseada. Usando las 
estrategias definidas en la sección anterior, este secuente se transforma en 
una secuencia de uno o más secuentes digamos 
$\G_1\vdash A_1;\ldots ;\G_k\vdash A_k$ siendo la nueva meta a 
resolver el secuente $\G_1\vdash A_1$, el cual genera nuevas submetas, y así 
sucesivamente. El proceso de búsqueda se simplifica con las siguientes 
definiciones:
\bi
\item $\Sc$ denota a una secuencia finita de metas (posiblemente vacía, 
denotada $\square$) 
$$ \Sc=_{def}\Ge_1;\ldots;\Ge_k $$

\item El proceso de búsqueda aplica una estrategia a la primera meta de la 
secuencia actual Si al aplicar cierta estrategia a la meta $\Ge_1$ se generan 
las submetas 
$\Ge'_{11};\Ge'_{12};\ldots;\Ge'_{1k}$ entonces escribimos
$$ \Ge_1;\Sc\;\rhd\;\Ge'_{11};\Ge;_{12};\ldots;\Ge_{1k};\Sc $$
y a este proceso le llamamos táctica.

\item La relación $\Sc\rhd \Sc'$ puede leerse como \enquote{para demostrar la 
secuencia~$\Sc$ es suficiente demostrar la secuencia~$\Sc'$}. Por ejemplo:
\bi
 \item Para demostrar que $p,q\vdash (q\lor r) \land p$ \\
  es suficiente demostrar que $p,q\vdash q\lor r$  y que $p,q\vdash p$
  por lo que escribimos 
  $p,q\vdash (q\lor r) \land p\;\rhd\;p,q\vdash q\lor r\;; \;p,q\vdash p$.
  
 \item Para demostrar que $p,q\vdash q\lor r$ y $p,q\vdash p$ \\
  es suficiente demostrar $p,q\vdash q$  y $p,q\vdash p$
  por lo que escribimos 
  $p,q\vdash q\lor r\; ; \;p,q\vdash p\;\rhd\; p,q\vdash q\; ; \;p,q\vdash p$.

 \item Para demostrar que $p,q\vdash q$ y $p,q\vdash p$ \\
 es suficiente demostrar $p,q\vdash p$ (pues $p,q\vdash q$ es inmediato)
por lo que escribimos 
$p,q\vdash q\; ; \;p,q\vdash p\;\rhd\;p,q\vdash p$.

 \item Para demostrar $p,q\vdash p$ \\
  hemos terminado pues $p,q\vdash p$ es inmediato por lo que escribimos 
  $p,q\vdash p\;\rhd\; \Box$ ,  donde el símbolo $\Box$ denota a la secuencia vacía de metas.
\ei
\ei

A continuación definimos las tácticas particulares. Aquí $\Sc$ denota a una 
secuencia arbitraria de metas y una expresión de la forma $H:A$ denota a 
una hipótesis etiquetada con el nombre $H$ el cual se usa como referencia en la 
definición de la táctica. En general un contexto tiene todas las hipótesis 
etiquetadas, es decir, es de la forma $\G=\{H_1:A_1,\ldots,H_n:A_n\}$.


\begin{itemize}
 \item \texttt{intro}: $\quad \G\vdash A\to B;\Sc\;\rhd\; \G,A\vdash B;\Sc$
\vspace*{5pt}
 \item \texttt{split}: 
 $\quad \G\vdash A\land B;\Sc\;\rhd\; \G\vdash A; \G\vdash B;\Sc$
\vspace*{5pt}
 \item \texttt{left}: $\quad \G\vdash A\lor B;\Sc\;\rhd\; \G\vdash A;\Sc$
\vspace*{5pt}
 \item \texttt{right}: $\quad \G\vdash A\lor B;\Sc\;\rhd\; \G\vdash B;\Sc$
\vspace*{5pt} 
 \item \texttt{intro}: 
 $\quad \G\vdash \forall x A;\Sc\;\rhd\;\G\vdash A;\Sc$ donde s.p.g 
 $x\notin FV(\G)$
\vspace*{5pt}
 \item \texttt{exists t}: 
 $\quad \G\vdash \exists x A;\Sc\;\rhd\;\G\vdash A[x:=t];\Sc$ para algún $t$.
\vspace*{5pt}
 \item \texttt{apply} H: 
 $\quad \G,H: A\to B;\G'\vdash B;\Sc\;\rhd\; \G,H:A\to B;\G'\vdash A;\Sc$
 \item \texttt{destruct} H: 
 $\quad \G,H:A\land B;\G'\vdash C;\Sc\;\rhd\; \G,H_1:A, H_2: B;\G'\vdash C;\Sc$
\vspace*{5pt}
 \item \texttt{destruct} H: 
 $\quad \G,H:A\lor B;\G'\vdash C;\Sc\;\rhd\;\G,H_1: A;\G'\vdash C\;\;;\;\;\G,H_2:B;\G'\vdash C;\Sc$
\vspace*{5pt}
 \item \texttt{apply} H: 
 $\quad \G, H:\forall xA;\G'\vdash A[x:=t];\Sc\;\rhd\;\Sc$
\vspace*{5pt}
 \item \texttt{destruct} H: 
 $\quad \G, H: \exists x A;\G'\vdash C;\Sc\;\rhd\; \G, H_1:A;\G'\vdash C;\Sc$
 donde s.p.g. 
  $x\notin FV(\G)$
\vspace*{5pt}
 \item \texttt{destruct} H \texttt{with} t: 
 $\quad \G, H: \fa x A;\G'\vdash C;\Sc\;\rhd\; \G, H:\fa x A, H_1 : A[x:=t];\G'\vdash C;\Sc$
\vspace*{5pt}
 \item \texttt{assumption}: $\quad \G,H:A\vdash A;\Sc \;\rhd\; \Sc$ 
 en particular $\G,A\vdash A\;\rhd\;\Box$
\vspace*{5pt}
 \item \texttt{assert} 
 $A$: $\quad \G\vdash C;\Sc\;\rhd\; \G\vdash A\;\;;\;\;\G,\;H:A\vdash C;\Sc$
	% \vspace*{5pt}
	% \item \texttt{cut}: $\G\vdash C;\Sc\;\rhd\; \G\vdash A\to C;\G\vdash 
% A;\Sc$
	% \vspace{0.3cm}
\end{itemize}

% Los nombres de las tácticas corresponden al nombre del comando en el 
%asistente 
% de prueba {\coq} visto en el laboratorio.


\medskip

Veamos algunos ejemplos de derivación mediante tácticas.
\bi
\item Probar que: \hspace{0.5cm} 
$\vdash (p \land q \to r) \to p \to q \to r$  
\[
\begin{array}{lr}
 \quad \vdash p \land q \to r \imp  p  \to q \to r &   
 intro\\  \vspace*{5pt}
 H_1: p \land q \to r \vdash  p  \to q \to r &  
 intro\\ \vspace*{5pt}
 H_1: p \land q \to r \; , \; H_2: p  \vdash q \to r & 
 intro \\ \vspace*{5pt}
 H_1: p \land q \to r \; , \; H_2: p \; , \; H_3: q \vdash  r &
 apply \;\; H_1\\ \vspace*{5pt}
 H_1: p \land q \to r \; , \; H_2: p \; , \; H_3: q \vdash p \land q &
 split \\ \vspace*{5pt}
 H_1: p \land q \to r \; , \; H_2: p \; , \; H_3: q \vdash p \; ; \; 
 H_1: p \land q \to r \; , \; H_2: p \; , \; H_3: q \vdash q & 
 assumption \\ \vspace*{5pt}
 H_1: p \land q \to r \; , \; H_2: p \; , \; H_3: q \vdash q & 
 assumption \\ \vspace*{5pt}
 \square & 
\end{array}
\]

\item Sea
$\G=\{H:p \to q \lor  r ,\; H': q\to r,\;H'': r\to s\}$. 
Queremos mostrar que $\G\vdash p\to s $ 
\[
\begin{array}{rll}
1\quad & \G\vdash p\to s &  \qquad \mbox{intro} \\
2 \quad & \G,\;H_1: p \vdash s & \qquad \mbox {apply}\; H'' \\ %\;\; r\to s \\
3 \quad & \G,\;H_1: p \vdash r & \qquad \mbox{assert} \;\; q \lor r \\
4 \quad & \G,\;H_1: p\vdash q\lor r\quad  ; \quad  \G,\;H_1: p,\;H_2: q\lor 
r\vdash r &  \qquad \mbox{apply}\; H \\ %\;p\to q\lor r \\
5 \quad & \G,\;H_1: p\vdash p\quad  ; \quad  \G,\;H_1: p,\;H_2: q\lor r\vdash 
r& \qquad \mbox{assumption} \\ 
6 \quad & \G,\;H_1: p,\;H_2: q\lor r\vdash r & \qquad \mbox{destruct} \;H_2 \\ 
%\;\; q\lor r \\
7\quad & \G,\;H_1: p,\;H_2: q \vdash r \quad  ; \quad  \G,\;H_1: p,\;H_3: r 
\vdash r &  \qquad \mbox{apply}\; H' \\ %\; q\to r \\ 
8 \quad & \G,\;H_1: p,\;H_2: q \vdash q \quad  ; \quad  \G,\;H_2: p,\;H_3: r 
\vdash r & \qquad \mbox{assumption} \\
9 \quad & \G,\;H_1: p,\;H_3: r \vdash r & \qquad  \mbox{assumption} \\
10 \quad & \square & 
\end{array}
\]
\ei


\subsection{Tácticas para la negación}
Las siguientes tácticas son útiles cuando hay que razonar con negación:

\begin{itemize}
%\item \texttt{exfalso}: $\;\;\G\;\vdash A \; ;\Sc\;\rhd\;\G\vdash \bot\; ;\; 
% \Sc$

 \item \texttt{absurd (A) : }  
  $\G \vdash B \, ; \, \Sc\;\rhd\;\;\G \vdash A \, \; ;\;\G\vdash \neg A\; ;\Sc$
 \item \texttt{contradict H: } 
  $\quad \G, H:\neg A\vdash B \,;\Sc\;\rhd\;\G\vdash A\; ;\Sc$
 \item \texttt{contradict H: } 
  $\quad \G, H:\neg A\vdash \neg B \,;\Sc\;\rhd\;\G, H:B\vdash A\; ;\Sc$
 \item \texttt{contradict H: } 
  $\quad \G, H: A\vdash B \, ;\Sc\;\rhd\;\G\vdash \neg A\; ;\Sc$
 \item \texttt{contradict H: } 
  $\quad \G, H: A\vdash \neg B \, ;\Sc\;\rhd\;\G, H:B\vdash \neg A\; ;\Sc$
\end{itemize}

Las siguientes tácticas sólo están disponibles en la lógica clásica al importar 
la biblioteca \texttt{Classical}
\bi
	% \item \texttt{contradict} H: $\quad \G, H:\neg A\vdash 
% B;\Sc\;\rhd\;\G\vdash A\; ;\Sc$
	% \item \texttt{contradict} H: $\quad \G, H:\neg A\vdash \neg 
% B;\Sc\;\rhd\;\G, H:B\vdash A\; ;\Sc$
	% \vspace*{5pt}
 \item \texttt{exact (classic (A))}: 
  $\quad \G\vdash A\lor\neg A\;;\Sc \;\rhd\;\Sc$
 \item \texttt{exact (NNPP (A))}: 
  $\quad \G\vdash \neg\neg A\to A\;;\Sc \;\rhd\;\Sc$
\ei

La primera de estas tácticas es útil en combinación con \texttt{assert} para 
agregar una instancia del tercero excluido al contexto y la segunda para 
agregar una instancia de la parte clásica de la ley de doble negación . Otras 
tácticas útiles derivada de éstas son:
\bi
 \item \texttt{destruct (classic (A))}: 
  $\quad \G\,\vdash B\;;\Sc\;\rhd\; \G,A\vdash B\;;\;\G,\neg A\vdash B;\;\Sc$.
 \item \texttt{apply (NNPP(A))}: 
  $\quad \G\vdash A\;;\;\Sc\;\rhd\;\G\vdash\neg\neg A\;;\;\Sc$
\ei


% \section{Los teoremas de completud y correctud para la lógica clásica}

% Finalizamos nuestras consideraciones acerca de los sistemas de deducción
% natural mencionando los teoremas de correctud y completud para la lógica
% clásica $\Dnc$ que vinculan el mundo de la semántica con el de la sintáxis 
% de manera biunívoca.

% \teo{[Correctud de $\Dnc$] Sean $\G$ un conjunto de fórmulas y $ A$ una
%   fórmula. La relación $\vdash_c$ es correcta con respecto a la
%   consecuencia lógica,es decir:
% \bc
% Si $\G\vdash_c A$ entonces $\G\models A$.
% \ec
% }
% \proof Inducción sobre $\G\vdash_c A$. Lo cual equivale a probar
% que todas las reglas del sistema $\Dnc$ preservan la noción $\models$ pero esto 
% es justo lo que dice la proposición 
% \ref{prop:rdnsem}.


% \teo{[Completud de $\Dnc$] Sean $\G$ un conjunto de fórmulas y $ A$ una
%   fórmula. El sistema $\Dnc$ es completo con respecto a la relación $\vdash_c$,
%   es decir:
% \bc
%  Si $\G\models A$ entonces $\G\vdash_c A$.
% \ec
% }
% \proof
% Omitimos la demostración dado que las técnicas necesarias pertenecen al
% ámbito de la lógica matemática pura.




% %Cabe observar que dado que los sistemas $\dnm$ y $\dnp$ son restricciones de
% %$\Dnc$ los teoremas recién enunciados son válidos haciendo las restricciones
% %necesarias a la noción de consecuencia lógica.

% %Ya hemos hecho las consideraciones generales acerca de los sistemas de
% %deducción natural en la nota cuatro, de manera que ahora sólo nos
% %limitaremos a presentar la extensión de los sistemas proposicionales
% %de deducción natural $\dnm,\dnp,\Dnc$ a la lógica de primer orden.


% %El sistema de deducción natural para la lógica {\bf clásica} de predicados,
% %consta de todas las reglas de la nota 4 junto con las reglas para los
% %cuantificadores recién definidas. Más adelante veremos las restricciones
% %minimal e intuicionista.

% % Obsérvese que las estrategias sirven sólo como guia, y puede que no siempre
% % sean el mejor camino para llegar a una conclusión. En particular no tenemos
% % una estrategia adecuada para probar existenciales directamente dado que según
% % las reglas la única manera posible de probar $\ex x A$ es probando
% % $ A[x:=t]$ para algún término $t$, pero dicho término tendría que adivinarse 
% % en general
% % puesto que desaparece en la conclusión de la regla $(\ex E)$. La mejor
% % estrategia en este caso es probar por contradicción, es decir:
% % \bi
% % \item Si $ A=\ex x B$ basta probar $\G,\neg\ex x B\vdash\bot$, o
% %   equivalentemente, usando las leyes de negación,
% % $\G,\fa x\neg B\vdash\bot$. Por supuesto esto prueba $\ex x  B$
% % sólo en la lógica clásica.
% % \ei

% \section{Reglas del sistema de Fitch (con cajas)}

% Una presentación común del sistema de deducción natural es usando el llamado 
% sistema de Fitch o sistema de cajas. Enunciamos aquí sus reglas por ser un 
% sistema importante, aunque no lo usaremos en este curso. En este sistema las 
% derivaciones son sucesiones de fórmulas y no de secuentes.
% \bi
% \item Conjunción:
% \begin{mathpar}
% \inferrule*[right=($\land$ I)]{A\quad \;\;B}{A\land B}

% \inferrule*[right=($\land$ E)]{A\land B}{A}

% \inferrule*[right=($\land$ E)]{A\land B}{B}
% \end{mathpar}

% \item Implicación:
% \begin{minipage}{.3\textwidth}
%  \centering
% \[
% \frac{
% \begin{tabular}{|c|}
% \hline
% A \\
% \vdots \\
% B \\
% \hline
% \end{tabular}}
% {A\to B}\;(\to I)
% \]
% \end{minipage}
% \begin{minipage}{.3\textwidth}
% \centering
% \begin{mathpar}
% \inferrule*[right=(MP)]{
% A\to B\quad \;\;A}{B}
% \end{mathpar}
% \end{minipage}


% \item Disyunción
% \begin{minipage}{.4\textwidth}
%  \centering
% \begin{mathpar}
% \inferrule*[right=($\lor$ I)]{A}{A\lor B}

% \inferrule*[right=($\lor$ I)]{B}{A\lor B}
% \end{mathpar}
% \end{minipage}
% \begin{minipage}{.4\textwidth}
% \centering
% \[
% \frac{\text{A}\lor \text{B}\quad 
% \begin{tabular}{|c|}
% \hline 
% A \\
% \vdots \\
% C \\
% \hline
% \end{tabular}\quad \;
% \begin{tabular}{|c|}
% \hline 
% B \\
% \vdots \\
% C \\
% \hline
% \end{tabular}
% }
% {\text{C}}\;(\lor E)
% \]
% \end{minipage}

% \item Verdad:
% \begin{mathpar}
% \inferrule*[right=($\top$ I)]{
% }{
% \quad\top\quad
% }
% \end{mathpar}

% \item Cuantificación universal:
% \[
% \frac{\quad 
% \begin{array}{|c|}
% \hline \\
% x_0\;\text{parámetro}\\ 
% \vdots \\
% A[x:=x_0]\\
% \\
% \hline
% \end{array}\quad 
% }{\fa x A}\;(\fa I)
% \]

% La condición de que $x_0$ sea un parámetro significa que $x_0$ no puede figurar 
% libre fuera de su caja.
% % \beqs
% % \inferrule*[]{
% % \begin{array}{|c|}
% % \hline
% % x_0\;\text{parámetro}\\
% % \vdots \\
% % A[x:=x_0] \\
% % \hline
% % \end{array}\quad \;\;
% % %x_0%\text{parámetro}
% % }{} %{\fa x\text{A}}\;(\fa I)
% % \eeqs
% \begin{mathpar}
% \inferrule*[right=($\fa$ E)]{\fa x\text{A}}
% {\text{A}[x:=t]}
% \end{mathpar}

% \item Cuantificación existencial:
% \begin{minipage}{.3\textwidth}
%  \centering
% \begin{mathpar}
% \inferrule*[right=($\ex$ I)]{A[x:=t]}{\ex x A}
% \end{mathpar}
% \end{minipage}
% \begin{minipage}{.3\textwidth}
% \centering
% \[
% \frac{\ex x A\quad \;\begin{array}{|c|}
% \hline \\
% A[x:=x_0] \\
% \vdots\\
% B\\
% \hline
% \end{array}\quad \;\;
% x_0\;\;\text{parámetro}}
% {B}\;(\ex E)
% \]
% \end{minipage}

% $x_0$ parámetro significa que $x_0$ no puede estar libre fuera de su caja, en 
% particular $x_0$ no debe estar libre en $B$.
% \ei


\end{document}
