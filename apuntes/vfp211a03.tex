\documentclass[11pt,letterpaper]{article}
%\usepackage{packageslc}
%\usepackage{optionslc}
\usepackage{amsthm,amssymb}
\usepackage[margin=1.8cm,includefoot]{geometry}
% \usepackage{lstcoq}
%\lstset{language=Haskell}
%\newcommand{\Q}{\ensuremath{\mathbb{Q}}}
\newcommand{\Z}{\ensuremath{\mathbb{Z}}}
\newcommand{\N}{\ensuremath{\mathbb{N}}}
\newcommand{\R}{\ensuremath{\mathbb{R}}}
%\newtheorem{lemma}{Lema}[section]
%\newtheorem{theorem}[lemma]{Teorema}
%\newtheorem{corollary}[lemma]{Corolario}
%\theoremstyle{definition}
%\newtheorem{definition}[lemma]{Definici\'on}
\newcommand{\A}{\mathcal{A}}
\newcommand{\E}{\ensuremath{\exists}}
\newcommand{\iso}{\ensuremath{\cong}}
\newcommand{\union}{\ensuremath{\cup}}
\newcommand{\morinyec}{\ensuremath{\precapprox}}
\newenvironment{prueba}{\vspace{-5mm}\noindent\textbf{Demostraci\'on}\\}{\noindent$\blacksquare$\\}
\newcommand{\nin}{\ensuremath{\notin}}
\newcommand{\tog}{\makebox[7mm][l]}
\newcommand{\toge}{\makebox[11mm][l]}
\newcommand{\toget}{\makebox[13mm][l]}
\newcommand{\togeth}{\makebox[14mm][l]}
\newcommand{\togethe}{\makebox[15mm][l]}
\newcommand{\together}{\makebox[17mm][l]}
\newcommand{\niso}{\ensuremath{\not \cong}}
\renewcommand\contentsname{\'Indice}
%\renewcommand\chaptername{Cap\'itulo}
\renewcommand\indexname{\'Indice}

%%\newcommand{\qed}{\hfill$\mathbb{Qed}$}
%\newcommand{\qed}{\hfill$\mathsf{\boldsymbol{\dashv}}$}


\newcommand{\Ejercicios}{\section*{Ejercicios}}

 \newenvironment{manitas}{%
      \renewcommand{\labelitemi}{\ding{44}}%
      \vspace{-0.5cm}%
      \begin{itemize}%
      \setlength{\itemsep}{0pt}\setlength{\parsep}{0pt}\setlength{\topsep}{0pt}%
      }{\end{itemize}}
\newenvironment{malitos}{%
      \renewcommand{\labelitemi}%
            {\raisebox{1.5ex}{\makebox[0.3cm][l]{\begin{rotate}{-90}%
            \ding{43}\end{rotate}}}}%
      \vspace{-0.5cm}%
      \begin{itemize}%
      \setlength{\itemsep}{0pt}\setlength{\parsep}{0pt}\setlength{\topsep}{0pt}%
      }{\end{itemize}}
\newenvironment{ejercs}{
     \renewcommand{\labelenumi}{\thesection.\theenumi.-}
     \renewcommand{\labelenumii}{\theenumii)}
     \begin{enumerate}}
     {\end{enumerate}}

   \newcommand{\bej}{\begin{ejercs}}
\newcommand{\eej}{\end{ejercs}}
   
%%=================================================================================

\newcommand{\bc}{\begin{center}}
\newcommand{\ec}{\end{center}}
\newcommand{\be}{\begin{enumerate}}
\newcommand{\ee}{\end{enumerate}}
\newcommand{\bi}{\begin{itemize}}
\newcommand{\ei}{\end{itemize}}
\newcommand{\beq}{\begin{equation}}
\newcommand{\eeq}{\end{equation}}
\newcommand{\beqs}{\begin{equation*}}
\newcommand{\eeqs}{\end{equation*}}
\newcommand{\ba}{\begin{array}}
\newcommand{\ea}{\end{array}}

\newcommand{\imp}{\rightarrow}
\newcommand{\Imp}{\Rightarrow}
\renewcommand{\iff}{\leftrightarrow}
\newcommand{\Iff}{\Leftrightarrow}
\newcommand{\G}{\Gamma}
\newcommand{\D}{\Delta}
\newcommand{\F}{\mathcal{F}}
\newcommand{\Ge}{\mathcal{G}}
\newcommand{\Pe}{\mathcal{P}}
\newcommand{\I}{\mathcal{I}}
\newcommand{\C}{\mathcal{C}}
\newcommand{\K}{\mathcal{K}}
\newcommand{\Kb}{\mathbb{K}}
\newcommand{\Eb}{\mathbb{E}}
\newcommand{\Ebs}{\mathbb{E}^\star}
\newcommand{\Ob}{\mathbb{O}}
\newcommand{\Ib}{\mathbb{I}}
\newcommand{\kb}{\bbkappa}
\newcommand{\M}{\mathcal{M}}
\newcommand{\Nc}{\mathcal{N}}
%\newcommand{\E}{\mathcal{E}}
%\newcommand{\R}{\mathcal{R}}
%\newcommand{\Q}{\mathcal{Q}}
\newcommand{\Sc}{\mathcal{S}}
\newcommand{\Sf}{\mathsf{\Sigma}}
\newcommand{\Te}{\mathcal{T}}
\newcommand{\Rb}{\mathbb{R}}
\newcommand{\Qb}{\mathbb{Q}}
\newcommand{\Kbb}{\mathbb{K}}
\newcommand{\T}{\mathbb{\Theta}}
\renewcommand{\L}{\mathcal{L}}
\newcommand{\fa}{\forall}
\newcommand{\ex}{\exists}
\newcommand{\inc}{\subseteq}
\newcommand{\lb}{\lambda}
\newcommand{\al}{\alpha}
\newcommand{\ga}{\gamma}
\newcommand{\Db}{\mathbb{D}}
\newcommand{\Fb}{\mathbb{F}}
\newcommand{\De}{\mathcal{D}}

\newcommand{\mg}{\mathbb{m}}

\newcommand{\cg}{\mathbb{C}}
\newcommand{\dg}{\mathbb{D}}
\newcommand{\jg}{\mathbb{J}}
\newcommand{\Ha}{\mathcal{H}}
%\newcommand{\A}{\mathcal{A}}
\newcommand{\sg}{\mathbb{S}}

\def\stackunder#1#2{\mathrel{\mathop{#2}\limits_{#1}}}



%\renewcommand{\qed}{\hfill$\boldsymbol{\dashv}$}
%\newcommand{\qed}{\hfill$\mathbb{Qed}$}

\newcommand{\pt}[1]{\langle #1 \rangle}



\renewcommand{\S}{\mathbb{\Sigma}}


\newcommand{\doubt}{\Red{{\LARGE {\sf ??}}}}

\newcommand{\coment}[1]{\hfill\\ \Big[{\bf Comentario Privado:} #1\Big]}
\newcommand{\preg}[1]{\hfill\\ \BrickRed{{\bf Pregunta:} #1}}
\newcommand{\conjet}[1]{\hfill\\ \OliveGreen{{\bf Conjecura:} #1}}






\newcommand{\pendiente}{\BrickRed{{\sc Pendiente}}}
\newcommand{\verifpendiente}{\BrickRed{{\sc Verificación pendiente}}}


\newcommand{\Mg}{\mathbb{M}}
\newcommand{\Bg}{\mathbb{B}}
\newcommand{\Lg}{\mathbb{L}}
\newcommand{\Tg}{\mathbb{T}}

\newcommand{\sketch}{\Red{{\sc sketch}}}




\newcommand{\B}{\mathbb{B}}
%\newcommand{\N}{\mathbb{N}}




%\newenvironment{leterize}{%
%        \renewcommand{\theenumi}{\alph{enumi}}
%        \begin{enumerate}}{\end{enumerate}}

%\newenvironment{manitas}{%
%      \renewcommand{\labelitemi}{\ding{44}}%
%      \vspace{-0.5cm}%
%      \begin{itemize}%
%      \setlength{\itemsep}{0pt}\setlength{\parsep}{0pt}\setlength{\topsep}{0pt}%
%      }{\end{itemize}}



\newcommand{\Lb}{\Lambda} 



\newcommand{\Om}{\Omega}

\newcommand{\W}{\mathcal{W}}


\newcommand{\Bc}{\mathcal{B}}
\newcommand{\Df}{\mathfrak{D}}
\newcommand{\Dc}{\mathcal{D}}
%\newcommand{\Tc}{\mathcal{T}}
\newcommand{\Mf}{\mathfrak{M}}

\newcommand{\Sg}{\mathbb{S}}

\newtheorem{theorem}{Teorema}
\newcommand{\teo}[1]{\begin{theorem} #1 \end{theorem}}
\newtheorem{proposition}{Proposici\'on}
\newcommand{\prop}[1]{\begin{proposition} #1 \end{proposition}}
\newtheorem{definition}{Definici\'on}
\newcommand{\defin}[1]{\begin{definition} #1 \end{definition}}
\newtheorem{corollary}{Corolario}
\newcommand{\cor}[1]{\begin{corollary} #1 \end{corollary}}
\newtheorem{lemma}{Lema}
\newcommand{\lema}[1]{\begin{lemma} #1 \end{lemma}}
\newcommand{\dem}[1]{\begin{proof} #1 \end{proof}}

%\renewcommand{\qed}{\qedsymbol{$\mathbf{\dashv}$}}

\newcommand{\proof}{\hfill\\\noindent\textbf{\textit{Demostraci\'on. }}}






\newcommand{\restr}[2]{#1\!\!\boldsymbol{\restriction}\!#2}



\newcommand{\vacio}{\varnothing}
\newcommand{\done}{\ensuremath{\checkmark}}

\newcommand{\ida}{$\Rightarrow \; )$ }
\newcommand{\regr}{$\Leftarrow \; )$ }


\newcommand{\ol}[1]{\overline{#1}}


\newcommand{\Tsf}{\mathsf{T}}

\newcommand{\inds}[1]{\index[simb]{#1}}



%--------------------------------------------------------------------------

\DeclareMathAlphabet{\mathpzc}{OT1}{pzc}{m}{it}

%\newcommand{\case}{\mathsf{case}}
%\renewcommand\labelitemi{$\circ$}
%%\newcommand{\qed}{\hfill$\mathbb{Qed}$}
\newcommand{\qed}{\hfill$\mathsf{\boldsymbol{\dashv}}$}

%\newcommand{\id}{\mathsf{Id}}

%\newcommand{\uc}{\mathcal{U}}
%\newcommand{\Ic}{\mathcal{I}}
%\newcommand{\pc}{\mathcal{P}}
%\newcommand{\qc}{\mathcal{Q}}
%\newcommand{\mc}{\mathcal{M}}
\newcommand{\supc}{\supseteq}
\newcommand{\limo}{\mathop{\mathpzc{Lim}}}
\newcommand{\ord}{\mathsf{OR}}

\newcommand{\hint}{\emph{Sugerencia: }}


\newcounter{EjempCtr}[section]
\newenvironment{enumrom}{\renewcommand{\theenumi}{\roman{enumi}}%
\renewcommand{\theenumii}{\roman{enumii}}
\renewcommand{\theenumiii}{\roman{enumiii}}
\renewcommand{\theenumiv}{\roman{enumiv}}
\begin{enumerate}}{\end{enumerate}}
\newenvironment{Ejemplo}
        {\stepcounter{EjempCtr}%
        \begin{description}\item[Ejemplo \thesection.\arabic{EjempCtr}]}%
        {\end{description}}
\newenvironment{demostr}{%
             {\em Demostración:}
                \begin{quotation}}{\end{quotation}}

   \newcommand{\beje}{\begin{Ejemplo}}
\newcommand{\eeje}{\end{Ejemplo}}

\newcommand{\propo}{\ensuremath{\mathsf{PROP}}}
\newcommand{\atom}{\ensuremath{\mathsf{ATOM}}}
\newcommand{\vphi}{\varphi}
\newcommand{\vp}{\varphi}

\newcommand{\dn}{\mathsf{DN}}
\newcommand{\dnC}{\mathsf{DN_C}}
\newcommand{\dnM}{\mathsf{DN_M}}
\newcommand{\dnp}{\mathsf{DN_p}}
\newcommand{\dnm}{\mathsf{DN_p^M}}
\newcommand{\dnc}{\mathsf{DN_p^C}}

\newtheorem{eje}{Ejemplo}[section]
\newcommand{\ejem}[1]{\begin{eje}\normalfont #1 \end{eje}}

\newcommand{\vx}{\vec{x}}
\newcommand{\vy}{\vec{y}}
\newcommand{\vz}{\vec{z}}
\newcommand{\vt}{\vec{t}}
\newcommand{\vf}{\vec{f}}
\newcommand{\term}{\ensuremath{\mathsf{TERM}}}
\newcommand{\form}{\mathsf{FORM}}

\newcommand{\true}{\mathop{\mathsf{true}}}
\newcommand{\espc}{\vspace{.3cm}}
%%% Local Variables: 
%%% mode: latex
%%% TeX-master: t
%%% End: 

%\newcommand{\scd}[2]{#1 \vdash #2}
\usepackage[spanish]{babel}
\newtheorem{theorem}{Theorem}

\title{Verificaci\'on Formal PCIC 2021-2 \\
Un pequeño ejemplo de distintos estilos de prueba.
}
% \title{Lógica Computacional 2020-2, nota de clase 15 \\
% Un cálculo de secuentes
% }
\author{Favio Ezequiel Miranda Perea \and %Araceli Liliana Reyes Cabello\and
Lourdes Del Carmen Gonz\'alez Huesca}  %\and Pilar Selene Linares Arévalo}
\date{\today}

\begin{document}
\maketitle
El texto está en inglés pues corresponde a un fragmento de un artículo sometido a publicación.

\begin{theorem}\label{thm:nn01v0}
  There does not exist a natural number $n$ such that $0<n<1$.
\end{theorem}
\begin{proof}
This is an immediate consequence of the well-ordering principle.
\end{proof}

The above can be considered as a traditional ``proof'' completely correct and 
acceptable from an advanced point of view. However, it does not provide enough 
elements to be mechanized or written as an actual rigorous proof.

Let us next present a textbook traditional proof.

\begin{theorem}[A textbook proof]\label{thm:nn01v1}
  There does not exist a natural number $n$ such that $0<n<1$.
\end{theorem}
\begin{proof}
 Let $S=\{n\in\mathbb{N}\mid 0<n<1\}$ and assume $S\neq\varnothing$. Take
 $r\in S$ to be the least element in $S$. Therefore $0<r<1$ which implies
 $0<r^2<r<1$. Thus $r^2\in S$, which contradicts the minimality of
 $r$. Therefore $S$ must be empty.
\end{proof}

This traditional textbook proof, though rigorous, is not suitable to be 
directly formalized in a Proof Assitant (PA), for apart of the implicit background knowledge 
required, it does not make explicit a train of thought useful to guide 
the mechanization. Our experiment consisted of mechanizing the above theorem 
according to different common strategies of reasoning. First we use forward 
reasoning, which means going exclusively from the hypotheses towards the 
conclusion.

\begin{theorem}[Forward proof]\label{thm:nn01vf}
  There does not exist a natural number $n$ such that $0<n<1$.
\end{theorem}
\begin{proof}
  We assume that $\exists n\in\mathbb{N}$ such that $0<n<1$.
  The well-ordering principle allow us to take a minimal $r\in\mathbb{N}$ such 
  that $0<r<1$. Since $0<r$ we get $0=0\cdot r< r\cdot r$ and
 as $r<1$ we obtain $r\cdot r<1\cdot r = r$. Thus $0<r^2<r$. From this, as
 $r<1$, we get $0<r^2<1$. The minimality of $r$
 implies now that $r\leq r^2$, which leads us to $r^2\not <r$, yielding a 
contradiction.
\end{proof}

Let us observe that in this proof there is no reference to the set~$S$ used in 
the proof of Theorem~\ref{thm:nn01v1}, for its use would complicate the 
mechanization in a unnecesary way. 
The proof looks similar to the previous one but it has an explicit forward 
structure and it provides us with more explanations. The proof script was 
constructed having Theorem~\ref{thm:nn01v1} as a guideline using forward 
reasoning exclusively. This means we avoid the native backward mechanisms of 
the PA, which resulted in an awkward computer-assisted proof.

Next we explore the opposite approach by constructing a proof using 
exclusively the backward strategy. The result is Theorem~\ref{thm:nn01vb} and 
was obtained discarding the previous proofs as a guideline, using only the same 
auxiliary results, namely the transitivity of the order relation, 
its compatibility with respect to the product and the relationship between the 
order relations % (lemmas~\ref{lem:opcomp},~\ref{lem:otrans} and~\ref{lem:nltleq} 
% below)
.
Since the example is quite elementary, this proof was constructed directly in 
the PA without resorting to pencil and paper.


\begin{theorem}[Backward proof]\label{thm:nn01vb}
  There does not exist a natural number $n$ such that $0<n<1$.
% Let $S=\{n\in\mathbb{N}\mid 0<n<1\}$. Then $S=\varnothing$  
\end{theorem}
\begin{proof}
  We assume that $\exists n\in\mathbb{N}$ such that $0<n<1$. Thus, the 
well-ordering principle allow us to take a minimal $r\in\mathbb{N}$ such that 
$0<r<1$.
We will arrive to a contradiction by showing that $r\cdot r \not < r$ and
  $r\cdot r < r$. To show $r\cdot r \not < r$ 
  it suffices to show $r\leq r\cdot r$. 
  To prove this, we use the minimality of $r$, and are required to show 
  $0<r\cdot r < 1$. 
  We separately prove the two inequalities: 
  the first one ($0<r\cdot r$) is a consequence of $0\cdot r< r\cdot r$, 
  which is solved by the known fact $0<r$. 
  For the second inequality it is enough to prove $r\cdot r < r < 1$. 
  Again, it suffices to prove the two inequalities separately. 
  The first one is a consequence of $r\cdot r < 1\cdot r$, which in turn is 
  entailed by the known fact $0<r<1$. 
  The second inequality $r<1$ is already known. 
  This finishes the proof of $r\cdot r\not < r$. 
  The remaining inequality $r\cdot r < r$ was already proven within the
  previous case. 
 \end{proof}

% en el script:
% bajo nivel: la llamada al principio del buen orden, la definicion de 
% negacion, la definicion de S como un predicado, la necesidad de probar False.
% el manejo de las desigualdades por separado, es decir ,a<b y b< c y no a<b<c
% la necesidad de probar lo mismo dos veces 

Theorem~\ref{thm:nn01vb} presents a verbose proof which, in comparison with the 
usual mathematical proof-writing style, has a cumbersome structure due to the 
exclusive use of backward reasoning. On the other hand, its mechanization is 
concise and more readable than the previous one.
% \nota{QUITAR: due to the balanced use of the PA facilities.} 
Let us also observe that the mechanization necessarily contains some mandatory 
low-level parts, like the need to repeat the proof (that is, the sequence of 
tactics) corresponding to show $r\cdot r < r$, which is undesirable from a 
programmer point of view. This and some other low-level issues can be 
partially prevented~\footnote{For instance by first proving 
$r\cdot r < r$ as an auxiliary lemma or at the beginning of the proof as in
Theorem~\ref{thm:nn01vbi}.}
but others like the explicit term rewriting of term~$r$ with term $1\cdot r$ 
are unavoidable. 

% Although both theorems~\ref{thm:nn01vf} and~\ref{thm:nn01vb} have mechanized 
% counterparts neither is a good candidate for  what we mean to be a transitional 
% proof. 
% In the first case the mechanized version is awkward, for the forward strategy 
% obliges us to use the same kind of tactic, namely {\tt assert}, all the 
% time. In the second case the traditional proof contains an informal detail in 
% order to avoid the repetition of a subproof, namely that of $r\cdot r<r$. 
% This part does not have an adequate counterpart in the succint mechanized 
% proof, 
% for instead of appealing to a result already proven in a previous subcase, we
% have to repeat the whole command sequence or to 
% invoke an auxiliary lemma twice.
% % \notita{The proof duplication could be replaced by an auxiliary lemma.}
% We also observe that in this case the traditional proof is cumbersome in 
% comparison with that in theorem~\ref{thm:nn01vf}.
% % In the second case the opposite happens, the mechanized proof is succint while 
% % its traditional counterpart results cumbersome.


Consider now the following proof, which was obtained by using 
Theorem~\ref{thm:nn01v1} as a guideline but also mantaining a full interaction 
with the PA.


\begin{theorem}[Bidirectional proof]\label{thm:nn01vbi}
  There does not exist a natural number $n$ such that $0<n<1$.
% Let $S=\{n\in\mathbb{N}\mid 0<n<1\}$. Then $S=\varnothing$  
\end{theorem}
\begin{proof}
  We assume that $\exists n\in\mathbb{N}$ such that $0<n<1$ and arrive to a 
contradiction.
  The well-ordering principle allow us to take a minimal $r\in\mathbb{N}$ such 
  that $0<r<1$. We first claim that $r\cdot r<r$. 
  Since  $0<r<1$, we get $r\cdot r<1\cdot r$, that is $r\cdot r<r$.
  Next, we explicitly contradict the previous claim by showing 
  $r\cdot r\not < r$. 
  To this purpose it suffices to show $r\leq r\cdot r$. From $0<r$ we get 
  $0=0\cdot r<r\cdot r$. Further, $r\cdot r<r$ and $r<1$ yield $r\cdot r<1$. 
  By the minimality of $r$, to show the required $r\leq r\cdot r$, it suffices 
  to show $0<r\cdot r<1$, but this has already been proven. 
  % To do so, we prove 
  % $0<r\cdot r<1$ and apply the minimality of $r$. 
  % Since $0<r$, then $0=0\cdot r<r\cdot r$. 
  % On the other hand we have $r\cdot r < r < 1$, which yields $r\cdot r < 1$.
\end{proof}


As expected this proof has the best of both worlds, namely a well organized 
text together with a concise proof script. This structure was obtained by a 
parallel construction of both proofs, one that adequately combines forward and 
backward reasoning. The full interaction with the PA helped to enhance the 
argumentation, in particular this proof lacks repeated subproofs, 
like the one for $r\cdot r<r$ in the proof of Theorem~\ref{thm:nn01vb}.


Finally, let us present a proof gained directly from a proof script whose motivation is to 
show a simpler mechanization than those already presented. One that does not 
use library theorems and which is certainly more succint than the previous ones 
Its simplicity lies on the use of the definitions of $<$ and $\leq$ 
given in the core library (automatically loaded when starting Coq). 


\begin{theorem}\label{thm:nn01vauto}
  There does not exist a natural number $n$ such that $0<n<1$.
\end{theorem}
\begin{proof}
  Assuming that there is a natural $n$ such that $0<n<1$ we will arrive to a contradiction. We have
  $0<n$ and $n<1$. According to the definition of $n<1$ we have $S\,n\leq 1$, which by definition yields two cases
  \begin{itemize}
  \item $S\,n=1$. Then we have $n=0$ and therefore $0<1$ and $0<0$, which yields the contradiction.
  \item $S\,n\leq 0$. This a direct contradiction.
  \end{itemize}
\end{proof}




\end{document}