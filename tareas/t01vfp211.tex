



\documentclass[spanish]{article}
\usepackage[T1]{fontenc}
\usepackage[latin9]{inputenc}
\usepackage{geometry}
\geometry{verbose,tmargin=1.5cm,lmargin=1.5cm,rmargin=1.5cm,headheight=1.5cm,headsep=1.5cm,footskip=1.5cm}
\usepackage[dvipsnames]{xcolor}
\usepackage{color}
\usepackage{calc}
\usepackage{hyperref}
\usepackage{graphicx}
\usepackage[document]{ragged2e}
%\usepackage{coloremoji}

\makeatletter

\providecommand{\tabularnewline}{\\}

\makeatother

\usepackage{babel}
\addto\shorthandsspanish{\spanishdeactivate{~<>.}}

\usepackage{listings}
\renewcommand{\lstlistingname}{Listado de c�digo}

\definecolor{mygray}{rgb}{0.240,0.240,0.240}
\definecolor{mygreen}{rgb}{0,0.6,0}
\definecolor{anti-flashwhite}{rgb}{0.95, 0.95, 0.96}
\definecolor{brightgreen}{rgb}{0.4, 1.0, 0.0}

\lstset{
 language=haskell,
 basicstyle={\ttfamily},
 columns=[c]fixed,
 breaklines=true,
 showstringspaces=false,
 stringstyle={\color{blue}\itshape},
 literate={'"'}{\textquotesingle "\textquotesingle}3,
 tabsize=3,
 otherkeywords={},
 keywordstyle={\color{mygreen}\bfseries},
 commentstyle={\color{gray}},
 backgroundcolor={\color{anti-flashwhite}},
 frame=single,
 numbers=left,
 rulecolor=\color{anti-flashwhite},
 numberstyle=\footnotesize
}


\begin{document}

\title{Verificaci�n Formal 2021-1\\
  PCIC IIMAS UNAM\\
Tarea 1}

\author{Favio E. Miranda Perea}

\date{ \today\\
Fecha de entrega: martes 20 de octubre de 2020, 2359hrs}

\maketitle

En cada caso debe entregarse la verificaci�n de forma tradicional y la mecanizaci�n en Coq.
Toda propiedad auxiliar debe tambi�n ser demostrada.
\\
Las pruebas tradicionales deben entregarse en pdf ya sea generadas en LaTex o a mano.
Las mecanizaciones deben entregarse en un archivo por cada ejercicio con el nombre ApellidoPaternoVfp211T01-k.v para el ejercicio $k$.

\bigskip

\begin{enumerate}
\item Desarrolle el caso de verificaci�n relacionado al aplanamiento de �rboles binarios y la propiedad de preservar la pertenencia de sus elementos.
\bigskip
\item Concluya el caso de equivalencia de las siguientes dos definiciones de �rden en naturales (
  demostrando todo lo admitido\footnote{{\tt Admitted} en el script.} en el script 2 de clase):
  \begin{enumerate}
  \item Definici�n 1:
    \[
      \frac{}{n\leq_1 n}(o1refl)\;\;\;\;\;\;\;\;\;\;\
      \frac{n\leq_1 m}{n\leq_1 S\,m}(o1s) 
    \]

    
    \[
      \frac{S\,n\leq_1 m}{n<_1m}(oe1)
      \]
    \item Definici�n 2:
    \[
      \frac{}{0 <_2 S\,n}(o2z)\;\;\;\;\;\;\;\;\;\;\
      \frac{n<_2 m}{S\,n <_2 S\,m}(o2s)
    \]

    \[
      n \leq_2 m \Leftrightarrow_{def} n = m \;\lor\;n <_2 m
      \]
  \end{enumerate}
\bigskip
\item Demuestre las siguientes propiedades que relacionan a la relaci�n de orden con el m�nimo y la suma. Use cualquiera de las definiciones de orden del ejercicio anterior.  
  \begin{enumerate}
  \item $n\leq m\to min\,n\,m = n$
  \item $m\leq n\to min\,n\,m = m$
  \item $n \leq n + m$
  \item $m \leq n + m$  
  \item $n\leq m \to n+p\leq m+p$
  \item $n \leq m \lor m \leq n$
  \item $n\leq m \leftrightarrow \exists p(m = n + p)$
  \end{enumerate}
%En caso de requerir propiedades adicionales, estas deben demostrarse tambi�n.
\bigskip
\item Demuestre la siguiente propiedad de la funci�n {\tt take} de listas:
  \begin{center}
    {\tt take} n . {\tt take} m = {\tt take} ({\tt min} m n) 
  \end{center}

\end{enumerate}  

% Chon Hacker desea que se implemente un int�rprete para el  siguiente lenguaje de expresiones de orden en n�meros naturales:
%   \[
%     e:= n \mid e<e \mid e\leq e
%   \]
% Para lo cual llama a dos compa�ias desarrolladoras.
%   \begin{enumerate}
%   \item Alima�a Software resuelve que el int�rprete se desarrollar� implementando de forma primitiva a la relaci�n $\leq$
%     y definiendo $<$ como azucar sint�ctica usando $\leq$, es decir como una definici�n usando $\leq$. La idea para implementar $n \leq m$ consiste en destruir a $m$ (es decir, ``restar uno'' o aplicar la funci�n predecesor, hasta que $n=m$ o bien $m=0$.
%    \item Little Duck Software resuelve que el int�rprete se llevar� a cabo como sigue:  se implementar� $<$ de forma primitiva y posteriormente se implementar� $\leq$ como azucar sint�ctica usando $<$. La idea para implementar $n<m$ consiste en destruir ambos $n$ y $m$ (aplicando el predecesor) hasta que alguno de los dos sea $0$ y decidir as� de manera directa. 
%    \end{enumerate}

% Antes de decidir qu� compa�ia llevar� a cabo el proyecto Chon Hacker desea verificar que las dos implementaciones son equivalentes. Ayude a Chon a verificar formalmente la equivalencia, razonando con definiciones inductivas. Si tiene �xito, Chon le otorgar� un puesto de ejecutivo ``V'' en su compa�ia.

   
   



\end{document}
