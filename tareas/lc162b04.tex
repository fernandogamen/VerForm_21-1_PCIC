\documentclass[11pt]{article}
\usepackage{amsmath,amssymb}
%\usepackage{amssymb,amsmath,stmaryrd}
\usepackage{mathrsfs}
\usepackage{epsfig}
\usepackage{verbatim}
%\usepackage{fancyheadings}
%\usepackage{anysize}
\usepackage[latin1]{inputenc}
\usepackage[spanish]{babel}
\newcommand{\Q}{\ensuremath{\mathbb{Q}}}
\newcommand{\Z}{\ensuremath{\mathbb{Z}}}
\newcommand{\N}{\ensuremath{\mathbb{N}}}
\newcommand{\R}{\ensuremath{\mathbb{R}}}
%\newtheorem{lemma}{Lema}[section]
%\newtheorem{theorem}[lemma]{Teorema}
%\newtheorem{corollary}[lemma]{Corolario}
%\theoremstyle{definition}
%\newtheorem{definition}[lemma]{Definici\'on}
\newcommand{\A}{\mathcal{A}}
\newcommand{\E}{\ensuremath{\exists}}
\newcommand{\iso}{\ensuremath{\cong}}
\newcommand{\union}{\ensuremath{\cup}}
\newcommand{\morinyec}{\ensuremath{\precapprox}}
\newenvironment{prueba}{\vspace{-5mm}\noindent\textbf{Demostraci\'on}\\}{\noindent$\blacksquare$\\}
\newcommand{\nin}{\ensuremath{\notin}}
\newcommand{\tog}{\makebox[7mm][l]}
\newcommand{\toge}{\makebox[11mm][l]}
\newcommand{\toget}{\makebox[13mm][l]}
\newcommand{\togeth}{\makebox[14mm][l]}
\newcommand{\togethe}{\makebox[15mm][l]}
\newcommand{\together}{\makebox[17mm][l]}
\newcommand{\niso}{\ensuremath{\not \cong}}
\renewcommand\contentsname{\'Indice}
%\renewcommand\chaptername{Cap\'itulo}
\renewcommand\indexname{\'Indice}

%%\newcommand{\qed}{\hfill$\mathbb{Qed}$}
%\newcommand{\qed}{\hfill$\mathsf{\boldsymbol{\dashv}}$}


\newcommand{\Ejercicios}{\section*{Ejercicios}}

 \newenvironment{manitas}{%
      \renewcommand{\labelitemi}{\ding{44}}%
      \vspace{-0.5cm}%
      \begin{itemize}%
      \setlength{\itemsep}{0pt}\setlength{\parsep}{0pt}\setlength{\topsep}{0pt}%
      }{\end{itemize}}
\newenvironment{malitos}{%
      \renewcommand{\labelitemi}%
            {\raisebox{1.5ex}{\makebox[0.3cm][l]{\begin{rotate}{-90}%
            \ding{43}\end{rotate}}}}%
      \vspace{-0.5cm}%
      \begin{itemize}%
      \setlength{\itemsep}{0pt}\setlength{\parsep}{0pt}\setlength{\topsep}{0pt}%
      }{\end{itemize}}
\newenvironment{ejercs}{
     \renewcommand{\labelenumi}{\thesection.\theenumi.-}
     \renewcommand{\labelenumii}{\theenumii)}
     \begin{enumerate}}
     {\end{enumerate}}

   \newcommand{\bej}{\begin{ejercs}}
\newcommand{\eej}{\end{ejercs}}
   
%%=================================================================================

\newcommand{\bc}{\begin{center}}
\newcommand{\ec}{\end{center}}
\newcommand{\be}{\begin{enumerate}}
\newcommand{\ee}{\end{enumerate}}
\newcommand{\bi}{\begin{itemize}}
\newcommand{\ei}{\end{itemize}}
\newcommand{\beq}{\begin{equation}}
\newcommand{\eeq}{\end{equation}}
\newcommand{\beqs}{\begin{equation*}}
\newcommand{\eeqs}{\end{equation*}}
\newcommand{\ba}{\begin{array}}
\newcommand{\ea}{\end{array}}

\newcommand{\imp}{\rightarrow}
\newcommand{\Imp}{\Rightarrow}
\renewcommand{\iff}{\leftrightarrow}
\newcommand{\Iff}{\Leftrightarrow}
\newcommand{\G}{\Gamma}
\newcommand{\D}{\Delta}
\newcommand{\F}{\mathcal{F}}
\newcommand{\Ge}{\mathcal{G}}
\newcommand{\Pe}{\mathcal{P}}
\newcommand{\I}{\mathcal{I}}
\newcommand{\C}{\mathcal{C}}
\newcommand{\K}{\mathcal{K}}
\newcommand{\Kb}{\mathbb{K}}
\newcommand{\Eb}{\mathbb{E}}
\newcommand{\Ebs}{\mathbb{E}^\star}
\newcommand{\Ob}{\mathbb{O}}
\newcommand{\Ib}{\mathbb{I}}
\newcommand{\kb}{\bbkappa}
\newcommand{\M}{\mathcal{M}}
\newcommand{\Nc}{\mathcal{N}}
%\newcommand{\E}{\mathcal{E}}
%\newcommand{\R}{\mathcal{R}}
%\newcommand{\Q}{\mathcal{Q}}
\newcommand{\Sc}{\mathcal{S}}
\newcommand{\Sf}{\mathsf{\Sigma}}
\newcommand{\Te}{\mathcal{T}}
\newcommand{\Rb}{\mathbb{R}}
\newcommand{\Qb}{\mathbb{Q}}
\newcommand{\Kbb}{\mathbb{K}}
\newcommand{\T}{\mathbb{\Theta}}
\renewcommand{\L}{\mathcal{L}}
\newcommand{\fa}{\forall}
\newcommand{\ex}{\exists}
\newcommand{\inc}{\subseteq}
\newcommand{\lb}{\lambda}
\newcommand{\al}{\alpha}
\newcommand{\ga}{\gamma}
\newcommand{\Db}{\mathbb{D}}
\newcommand{\Fb}{\mathbb{F}}
\newcommand{\De}{\mathcal{D}}

\newcommand{\mg}{\mathbb{m}}

\newcommand{\cg}{\mathbb{C}}
\newcommand{\dg}{\mathbb{D}}
\newcommand{\jg}{\mathbb{J}}
\newcommand{\Ha}{\mathcal{H}}
%\newcommand{\A}{\mathcal{A}}
\newcommand{\sg}{\mathbb{S}}

\def\stackunder#1#2{\mathrel{\mathop{#2}\limits_{#1}}}



%\renewcommand{\qed}{\hfill$\boldsymbol{\dashv}$}
%\newcommand{\qed}{\hfill$\mathbb{Qed}$}

\newcommand{\pt}[1]{\langle #1 \rangle}



\renewcommand{\S}{\mathbb{\Sigma}}


\newcommand{\doubt}{\Red{{\LARGE {\sf ??}}}}

\newcommand{\coment}[1]{\hfill\\ \Big[{\bf Comentario Privado:} #1\Big]}
\newcommand{\preg}[1]{\hfill\\ \BrickRed{{\bf Pregunta:} #1}}
\newcommand{\conjet}[1]{\hfill\\ \OliveGreen{{\bf Conjecura:} #1}}






\newcommand{\pendiente}{\BrickRed{{\sc Pendiente}}}
\newcommand{\verifpendiente}{\BrickRed{{\sc Verificación pendiente}}}


\newcommand{\Mg}{\mathbb{M}}
\newcommand{\Bg}{\mathbb{B}}
\newcommand{\Lg}{\mathbb{L}}
\newcommand{\Tg}{\mathbb{T}}

\newcommand{\sketch}{\Red{{\sc sketch}}}




\newcommand{\B}{\mathbb{B}}
%\newcommand{\N}{\mathbb{N}}




%\newenvironment{leterize}{%
%        \renewcommand{\theenumi}{\alph{enumi}}
%        \begin{enumerate}}{\end{enumerate}}

%\newenvironment{manitas}{%
%      \renewcommand{\labelitemi}{\ding{44}}%
%      \vspace{-0.5cm}%
%      \begin{itemize}%
%      \setlength{\itemsep}{0pt}\setlength{\parsep}{0pt}\setlength{\topsep}{0pt}%
%      }{\end{itemize}}



\newcommand{\Lb}{\Lambda} 



\newcommand{\Om}{\Omega}

\newcommand{\W}{\mathcal{W}}


\newcommand{\Bc}{\mathcal{B}}
\newcommand{\Df}{\mathfrak{D}}
\newcommand{\Dc}{\mathcal{D}}
%\newcommand{\Tc}{\mathcal{T}}
\newcommand{\Mf}{\mathfrak{M}}

\newcommand{\Sg}{\mathbb{S}}

\newtheorem{theorem}{Teorema}
\newcommand{\teo}[1]{\begin{theorem} #1 \end{theorem}}
\newtheorem{proposition}{Proposici\'on}
\newcommand{\prop}[1]{\begin{proposition} #1 \end{proposition}}
\newtheorem{definition}{Definici\'on}
\newcommand{\defin}[1]{\begin{definition} #1 \end{definition}}
\newtheorem{corollary}{Corolario}
\newcommand{\cor}[1]{\begin{corollary} #1 \end{corollary}}
\newtheorem{lemma}{Lema}
\newcommand{\lema}[1]{\begin{lemma} #1 \end{lemma}}
\newcommand{\dem}[1]{\begin{proof} #1 \end{proof}}

%\renewcommand{\qed}{\qedsymbol{$\mathbf{\dashv}$}}

\newcommand{\proof}{\hfill\\\noindent\textbf{\textit{Demostraci\'on. }}}






\newcommand{\restr}[2]{#1\!\!\boldsymbol{\restriction}\!#2}



\newcommand{\vacio}{\varnothing}
\newcommand{\done}{\ensuremath{\checkmark}}

\newcommand{\ida}{$\Rightarrow \; )$ }
\newcommand{\regr}{$\Leftarrow \; )$ }


\newcommand{\ol}[1]{\overline{#1}}


\newcommand{\Tsf}{\mathsf{T}}

\newcommand{\inds}[1]{\index[simb]{#1}}



%--------------------------------------------------------------------------

\DeclareMathAlphabet{\mathpzc}{OT1}{pzc}{m}{it}

%\newcommand{\case}{\mathsf{case}}
%\renewcommand\labelitemi{$\circ$}
%%\newcommand{\qed}{\hfill$\mathbb{Qed}$}
\newcommand{\qed}{\hfill$\mathsf{\boldsymbol{\dashv}}$}

%\newcommand{\id}{\mathsf{Id}}

%\newcommand{\uc}{\mathcal{U}}
%\newcommand{\Ic}{\mathcal{I}}
%\newcommand{\pc}{\mathcal{P}}
%\newcommand{\qc}{\mathcal{Q}}
%\newcommand{\mc}{\mathcal{M}}
\newcommand{\supc}{\supseteq}
\newcommand{\limo}{\mathop{\mathpzc{Lim}}}
\newcommand{\ord}{\mathsf{OR}}

\newcommand{\hint}{\emph{Sugerencia: }}


\newcounter{EjempCtr}[section]
\newenvironment{enumrom}{\renewcommand{\theenumi}{\roman{enumi}}%
\renewcommand{\theenumii}{\roman{enumii}}
\renewcommand{\theenumiii}{\roman{enumiii}}
\renewcommand{\theenumiv}{\roman{enumiv}}
\begin{enumerate}}{\end{enumerate}}
\newenvironment{Ejemplo}
        {\stepcounter{EjempCtr}%
        \begin{description}\item[Ejemplo \thesection.\arabic{EjempCtr}]}%
        {\end{description}}
\newenvironment{demostr}{%
             {\em Demostración:}
                \begin{quotation}}{\end{quotation}}

   \newcommand{\beje}{\begin{Ejemplo}}
\newcommand{\eeje}{\end{Ejemplo}}

\newcommand{\propo}{\ensuremath{\mathsf{PROP}}}
\newcommand{\atom}{\ensuremath{\mathsf{ATOM}}}
\newcommand{\vphi}{\varphi}
\newcommand{\vp}{\varphi}

\newcommand{\dn}{\mathsf{DN}}
\newcommand{\dnC}{\mathsf{DN_C}}
\newcommand{\dnM}{\mathsf{DN_M}}
\newcommand{\dnp}{\mathsf{DN_p}}
\newcommand{\dnm}{\mathsf{DN_p^M}}
\newcommand{\dnc}{\mathsf{DN_p^C}}

\newtheorem{eje}{Ejemplo}[section]
\newcommand{\ejem}[1]{\begin{eje}\normalfont #1 \end{eje}}

\newcommand{\vx}{\vec{x}}
\newcommand{\vy}{\vec{y}}
\newcommand{\vz}{\vec{z}}
\newcommand{\vt}{\vec{t}}
\newcommand{\vf}{\vec{f}}
\newcommand{\term}{\ensuremath{\mathsf{TERM}}}
\newcommand{\form}{\mathsf{FORM}}

\newcommand{\true}{\mathop{\mathsf{true}}}
\newcommand{\espc}{\vspace{.3cm}}
%%% Local Variables: 
%%% mode: latex
%%% TeX-master: t
%%% End: 


%\marginsize{2cm}{2cm}{2cm}{2cm}
%\pagestyle{fancyplain}

\usepackage[margin=2.5cm,includefoot]{geometry}

% \newcommand{\asc}[2]{#1\;\mathsf{as}\;#2}
% \newcommand{\inte}{\mathsf{int}}
% \newcommand{\float}{\mathsf{float}}

%\input{deflpp81}
%\newtheorem{defin}{Definici\'on}
\newtheorem{teo}{Teorema}
\newtheorem{lema}{Lema}
\newtheorem{coro}{Corolario}
\newtheorem{prop}{Proposici\'on}
%\newtheorem*{proof}{Demostraci\'on}
%\newcommand{\proof}{\hfill\\\noindent\textbf{\textit{Demostraci\'on. }}}
%\newcommand{\qed}{\hfill$\mathsf{\boldsymbol{\dashv}}$}
\newtheorem{eje}{Ejemplo}[section]
\newcommand{\ejem}[1]{\begin{eje}\normalfont #1 \end{eje}}      


\newcounter{EjempCtr}[section]
% \newenvironment{enumrom}{\renewcommand{\theenumi}{\roman{enumi}}%
% \renewcommand{\theenumii}{\roman{enumii}}
% \renewcommand{\theenumiii}{\roman{enumiii}}
% \renewcommand{\theenumiv}{\roman{enumiv}}
% \begin{enumerate}}{\end{enumerate}}
\newenvironment{Ejemplo}
        {\stepcounter{EjempCtr}%
        \begin{description}\item[Ejemplo \thesection.\arabic{EjempCtr}]}%
        {\end{description}}



\newcommand{\bc}{\begin{center}}
\newcommand{\ec}{\end{center}}
\newcommand{\be}{\begin{enumerate}}
\newcommand{\ee}{\end{enumerate}}
\newcommand{\bi}{\begin{itemize}}
\newcommand{\ei}{\end{itemize}}
\newcommand{\bd}{\begin{description}} 
\newcommand{\ed}{\end{description}}
\newcommand{\beq}{\begin{equation}}
\newcommand{\eeq}{\end{equation}}
\newcommand{\beqs}{\begin{equation*}}
\newcommand{\eeqs}{\end{equation*}}
\newcommand{\ba}{\begin{array}}
\newcommand{\ea}{\end{array}}
\newcommand{\bpt}{\begin{parsetree}}
\newcommand{\ept}{\end{parsetree}}
\newcommand{\bt}{\begin{tabular}}
\newcommand{\et}{\end{tabular}}
\newcommand{\bal}{\begin{align*}}
\newcommand{\eal}{\end{align*}}
\newcommand{\bv}{\begin{verbatim}}
\newcommand{\ev}{\end{verbatim}}


\newcommand{\fa}{\forall}
\newcommand{\ex}{\exists}



\newcommand{\pt}[1]{\langle #1 \rangle}
\newcommand{\espc}{\vspace{0.5cm}}
\newcommand{\pl}{{\sc Prolog }}

\newcommand{\imp}{\rightarrow}
\newcommand{\Imp}{\Rightarrow}
\newcommand{\sii}{\leftrightarrow}
\newcommand{\Sii}{\Leftrightarrow}
\newcommand{\syss}{\leftrightarrow}

\newcommand{\vp}{\varphi}
\newcommand{\cv}{\varepsilon}

\newcommand{\Q}{\ensuremath{\mathbb{Q}}}
\newcommand{\Z}{\ensuremath{\mathbb{Z}}}
\newcommand{\N}{\ensuremath{\mathbb{N}}}
\newcommand{\R}{\ensuremath{\mathbb{R}}}
\newcommand{\Pe}{\mathcal{P}}
\newcommand{\vacio}{\varnothing}
\newcommand{\inc}{\subseteq}


\newcommand{\ala}{\mathsf{ala}}
\newcommand{\asa}{\mathsf{asa}}

\newcommand{\suma}{\mathop{\mathtt{suma}}}
\renewcommand{\prod}{\mathop{\mathtt{prod}}}
\newcommand{\num}{\mathop{\mathtt{num}}}
\newcommand{\bol}{\mathop{\mathtt{bol}}}
\newcommand{\bool}{\mathsf{Bool}}
\newcommand{\lete}[2]{\mathtt{let}\;#1\;\mathtt{in}\;#2\;\mathtt{end}}
\newcommand{\leta}{\mathtt{let}}
\newcommand{\letrec}[5]{\mathsf{letrec}\;#1(#2):#3\;\Imp\;#4\;\mathsf{in}\;#5}


\newcommand{\ok}{\mathsf{ok}}

\newcommand{\true}{\mathop{\mathsf{true}}}
\newcommand{\false}{\mathop{\mathsf{false}}}
\newcommand{\ifte}[3]{\mathsf{if\;}#1\mathsf{\; then\;}#2\mathsf{\;
    else\;}#3}

\newcommand{\nat}{\mathop{\mathsf{Nat}}}
\newcommand{\suc}{\mathop{\mathsf{suc}}}
\newcommand{\pred}{\mathop{\mathsf{pred}}}
\newcommand{\iszero}{\mathop{\mathsf{iszero}}}

\newcommand{\afn}{\mathtt{lam}}
\newcommand{\fn}[2]{\mathsf{fun}(#1)\;\Imp\; #2}
\newcommand{\fnf}[4]{\mathsf{fun}\;#1\,(#2):#3\;\Imp\; #4}
\newcommand{\fnv}[2]{\mathsf{fun}(\mathsf{val}\;#1)\;\Imp\; #2}
\newcommand{\fnn}[2]{\mathsf{fun}(\mathsf{name}\;#1)\;\Imp\; #2}
\newcommand{\app}{\mathtt{app}}
\newcommand{\letn}[2]{\mathsf{let\; name}\;#1\;\mathsf{in}\;#2\;\mathsf{end}}
\newcommand{\letv}[2]{\mathsf{let\; val}\;#1\;\mathsf{in}\;#2\;\mathsf{end}}


\newcommand{\lista}{\mathop{\mathsf{list}}}

\newcommand{\arbol}{\mathop{\mathsf{tree}}}
\newcommand{\hoja}[1]{\mathsf{leaf}(#1)}
\newcommand{\nodo}[3]{\mathsf{node}(#1,#2,#3)}


\newcommand{\pila}{\mathsf{pila}}
\newcommand{\mco}{\;\mathsf{marco}}
\newcommand{\Pz}{\mathpzc{P}}
\newcommand{\mz}{\mathpzc{m}}
\newcommand{\pv}{\square}
\newcommand{\kred}{\longrightarrow_\K\;}

\newcommand{\est}{\mathsf{estado}}
\newcommand{\ini}{\mathsf{inicial}}
\newcommand{\fin}{\mathsf{final}}


\newcommand{\cref}{\mathop{\mathsf{ref}}}
\newcommand{\tref}{\mathop{\mathsf{Ref}}}
\newcommand{\pcf}{\ensuremath{\mathsf{PCF}\;}}

\newcommand{\error}{\mathsf{error}}


\newcommand{\catch}{\mathtt{catch}}
\newcommand{\pprec}{\prec\!\!\!\prec}

\newcommand{\rai}{\mathtt{raise}}
\newcommand{\handle}{\mathtt{handle}}

\newcommand{\texn}{\mathsf{TExn}}

\newcommand{\letcc}[2]{\mathsf{letcc}\;#1\;\mathsf{in}\;#2\;\mathsf{end}}

\newcommand{\continue}[2]{\mathsf{continue}\;#1\;#2}

\newcommand{\lcc}{\mathtt{letcc}}
\newcommand{\conti}{\mathtt{continue}}
\newcommand{\cont}{\mathtt{cont}}
\newcommand{\Cont}{\mathsf{Cont}}

\newcommand{\sbck}[1]{[\![#1]\!]}

\newcommand{\pmi}{\leftarrow}
\newcommand{\si}{\sigma}
%\newcommand{\cv}{\square}
\newcommand{\impp}{:-}

\renewcommand{\P}{\mathsf{P}}
\newcommand{\Tf}{\mathsf{T}}
\newcommand{\Sf}{\mathsf{S}}
\newcommand{\Rf}{\mathsf{R}}

\newcommand{\Lb}{\Lambda} 
\newcommand{\lb}{\lambda}
\newcommand{\al}{\alpha}
\newcommand{\ga}{\gamma}
\newcommand{\Db}{\mathbb{D}}
\newcommand{\Fb}{\mathbb{F}}
\newcommand{\De}{\mathcal{D}}

\newcommand{\D}{\mathsf{D}}
\renewcommand{\N}{\mathsf{N}}
\newcommand{\T}{\mathsf{T}}
\newcommand{\E}{\mathsf{E}}
\newcommand{\F}{\mathsf{F}}
\newcommand{\X}{\mathsf{X}}
\newcommand{\V}{\mathsf{V}}

\newcommand{\id}{\mathsf{Id}}
%\newcommand{\betared}{\imp_\beta}
\newcommand{\lbu}{\lambda^\mathsf{U}}
\newcommand{\dom}{\mathop{\mathsf{dom}}}


\newcommand{\tnat}{\mathsf{Nat}}
\newcommand{\tbool}{\mathsf{Bool}}
\newcommand{\asuc}{\mathop{\mathtt{suc}}}
\newcommand{\apred}{\mathop{\mathtt{pred}}}
\newcommand{\abool}{\mathop{\mathtt{bool}}}
\newcommand{\aiszero}{\mathop{\mathtt{iszero}}}
\newcommand{\aif}{\mathtt{if}}
\newcommand{\val}{\mathsf{valor}}


\newcommand{\tlist}{\mathsf{List}}
\newcommand{\isnil}{\mathop{\mathsf{isnil}}}
\newcommand{\head}{\mathop{\mathsf{head}}}
\newcommand{\tail}{\mathop{\mathsf{tail}}}
\newcommand{\nil}{\mathop{\mathsf{nil}}}
\newcommand{\cons}{\mathop{\mathsf{cons}}}


\newcommand{\G}{\Gamma}

\newcommand{\betared}{\to_\beta}

%\newcommand{\pcf}{\ensuremath{\mathsf{PCF}}}
\renewcommand{\Sf}{\mathsf{S}}
%\newcommand{\bool}{\mathsf{Bool}}
\newcommand{\cero}{\mathtt{cero}}

\newcommand{\recp}[5]{\mathsf{rec}\;#1\;\{0\Imp #2\;|\;s(#3)\;\mathsf{with}\;#4 \Imp #5\}}
\newcommand{\arecp}{\mathtt{rec}}
\newcommand{\fix}[2]{\mathsf{fix}\,#1\Imp\,#2}
\newcommand{\afix}{\mathtt{fix}}        
\newcommand{\mhs}{{\sf MinHs }}
\newcommand{\fst}{\mathop{\mathsf{fst}}}
\newcommand{\snd}{\mathop{\mathsf{snd}}}
\newcommand{\afst}{\mathop{\mathtt{fst}}}
\newcommand{\asnd}{\mathop{\mathtt{snd}}}
\newcommand{\unit}{\mathtt{Void}}
\newcommand{\inl}{\mathop{\mathsf{inl}}}
\newcommand{\inr}{\mathop{\mathsf{inr}}}
\newcommand{\ainl}{\mathop{\mathtt{inl}}}
\newcommand{\ainr}{\mathop{\mathtt{inr}}}


\newcommand{\rcd}{\mathtt{rcd}}
\newcommand{\prj}{\mathtt{prj}}
\newcommand{\abt}{\mathsf{abort}}
\newcommand{\aabt}{\mathtt{abort}}
\newcommand{\case}[5]{\mathsf{case}\;#1\;\mathsf{of}\;\{\inl\,#2\;\Imp\;#3\;|\;\inr\,#4\;\Imp\;#5\}}
\newcommand{\acase}{\mathtt{case}}
\newcommand{\void}{\mathsf{void}}


\newcommand{\qn}{\mathop{\mathsf{qn}}}
\newcommand{\pn}{\mathop{\mathsf{pn}}}
\newcommand{\sft}{\mathop{\mathsf{shift}}}


\renewcommand{\P}{\mathsf{P}}
\renewcommand{\Sf}{\mathsf{S}}
%\newcommand{\cref}{\mathop{\mathsf{ref}}}
%\newcommand{\tref}{\mathop{\mathsf{Ref}}}
%\newcommand{\letcc}[2]{\mathsf{letcc}\;#1\;\mathsf{in}\;#2\;\mathsf{end}}

%\newcommand{\continue}[2]{\mathsf{continue}\;#1\;#2}
%\newcommand{\Cont}{\mathsf{Cont}}
\DeclareMathAlphabet{\mathpzc}{OT1}{pzc}{m}{it}


\newcommand{\asdef}{:=_{def}}
%\newcommand{\error}{\mathtt{error}}
%\newcommand{\catch}{\mathtt{catch}}
%\newcommand{\Pz}{\mathpzc{P}}


%\DeclareMathAlphabet{\mathpzc}{OT1}{pzc}{m}{it}
\newcommand{\K}{\mathcal{K}}
\renewcommand{\L}{\mathcal{L}}
\renewcommand{\S}{\Sigma}


\newcommand{\asc}[2]{\pt{#2}\;#1}

%\newcommand{\asc}[2]{#1\;\mathsf{as}\;#2}

\newcommand{\inte}{\mathsf{Int}}
\newcommand{\float}{\mathsf{Float}}

\newcommand{\efe}{\mathsf{F}}
\newcommand{\lbi}{\lb^\imp}

\newcommand{\Xf}{\mathsf{X}}
\newcommand{\Yf}{\mathsf{Y}}
\newcommand{\Zf}{\mathsf{Z}}
%\DeclareMathAlphabet{\mathpzc}{OT1}{pzc}{m}{it}
%\newcommand{\mz}{\mathpzc{m}}
%\renewcommand{\Sf }{\Sf igma}


\newcommand{\Fn}[2]{\mathsf{FunP}\;(#1)\;\Imp\; #2}

\newcommand{\eqdef}{=_{def}}


\newcommand{\fls}{\mathsf{fields}}
\newcommand{\mty}{\mathsf{mtype}}
\newcommand{\mby}{\mathsf{mbody}}

%\DeclareMathAlphabet{\mathpzc}{OT1}{pzc}{m}{it}
%\newcommand{\mz}{\mathpzc{m}}
\renewcommand{\S}{\Sigma}


\newcommand{\nw}{\mathtt{new}}
\newcommand{\Bl}{\mathtt{B}}
\newcommand{\Cl}{\mathtt{C}}
\newcommand{\Dl}{\mathtt{D}}
\newcommand{\CL}{\mathtt{CL}}
\newcommand{\Kt}{\mathtt{K}}
\newcommand{\spr}{\mathtt{super}}
\newcommand{\ths}{\mathtt{this}}
\newcommand{\Mt}{\mathtt{M}}
\newcommand{\mt}{\mathtt{m}}
\newcommand{\rtn}{\mathtt{return}}
\newcommand{\extd}{\mathtt{extends}}
\newcommand{\cls}{\mathtt{class}}


\newcommand{\Int}{\mathsf{Int}}
\newcommand{\String}{\mathsf{String}}


\newcommand{\pack}{\mathtt{pack}}
\newcommand{\open}{\mathtt{open}}

\newcommand{\pair}{\mathsf{pair}}

\newcommand{\enough}{{\;\;\rhd\;\;}}

\title{L\'ogica Computacional 2016-2\\
Bolet\'in de ejercicios 4}
\author{Profesor: Dr. Favio Ezequiel Miranda Perea \\ Ayudante teor\'ia: Susana Hahn\\ Facultad de Ciencias UNAM}
\date{\today}


\begin{document}
\maketitle
	
%\noindent {\bf Sistemas de deducci\'on natural}
\begin{enumerate}
\item Decidir mediante deducci\'on natural.
\begin{itemize}
\item $P \to Q, Q \to R \vdash P \to Q \land R$
\item $P \to Q \vdash (Q \to R) \to (P \to R)$
\item $(A \lor (B \to A)) \land B \vdash A$
\item $A \to B \vdash (A \land B \to A) \land (A \to A \lor B)$
\item $A \land (B \lor C) \vdash (A \land B) \lor (A \land C)$
\item $\vdash_{m} \lnot \lnot (A \land B) \to \lnot \lnot A \land \lnot \lnot B $
\item $\vdash_{m} \lnot \lnot \forall x A \to \forall x \lnot \lnot A$
\item $\vdash_{i} ((P \to Q)\to P) \to P$
\item $\forall x(P x \to \exists y Qxy \to \exists yRyx), \exists xQax \vdash_{i}
\exists x(P x \to \exists yRyx)$
\item $P \to Q, \lnot Q \vdash_{i} \lnot P$
\item $L \land M \to P, I \to P, M, I \vdash_{i} \lnot L$
\item $P \lor Q, \lnot P \vdash_{i} Q$
\item $A \leftrightarrow B \vdash (A \land B) \lor (\lnot A \land \lnot B)$
\item $P \to Q, \lnot Q \vdash_{m} \lnot P$
\item $Pa , Qa \vdash \exists x(Px \land Qx) $ 

\end{itemize}
%\end{enumerate}





\item Realizar las siguientes derivaciones en l�gica {\bf
    minimal}.
  \be
 % \item $\vdash \neg\neg(A\lor\neg A)$
  \item $\vdash \neg\neg\neg A \iff \neg A$
  \item      $\vdash \neg\neg(A\land B)\imp \neg\neg
   A\land\neg\neg B$
  \item $\neg\neg\fa x A\imp \fa x \neg\neg A$
   \ee
 \item En este ejercicio se pide verificar la validez de la
   equivalencia l�gica \beqs\neg\neg(A\imp B)\iff \neg\neg A\imp\neg\neg
   B\eeqs en la l�gica {\bf intuicionista}, de la siguiente manera:
   \be
    \item $\vdash_{m} \neg\neg(A\imp B)\imp\neg\neg A\imp\neg\neg
   B$. \\ \emph{Sugerencia:} pruebe previamente y utilice el lema: 
   $\neg\neg A,\neg B\vdash \neg (A\imp B)$
       \item $\vdash_{i}(\neg\neg A\imp\neg\neg
   B)\imp\neg\neg(A\imp B)$. \\ \emph{Sugerencia:} primero construya
   derivaciones de $\neg\neg A$ y $\neg B$ usando $\neg(A\imp B)$ como
   hip�tesis.
    \ee


\item Realizar las siguientes derivaciones indicando el sistema usado
  $\vdash_m,\vdash_i$ � $\vdash_c$.
  \bi
%  \item $\neg r\lor\neg s, p\imp r, B\imp s\vdash\neg p\lor\neg B$
%    \item $p\imp B\lor r, B\imp r, r\imp s\vdash p\imp s$
\item $A\imp B, \neg(A\lor C)\imp D, A\lor B\imp C\vdash \neg D\imp C$
%       \item $\vdash (\neg p\imp p)\imp p$
%         \item $\vdash (\neg p\imp\neg B)\imp (\neg p\imp B)\imp p$
   %       \item $\vdash ((p\imp B)\imp p)\imp p$
\item $A\lor B\vdash (\neg D\imp C)\land (\neg B\imp\neg A)\imp
  (C\imp\neg B)\imp D$
      \item $\fa x(Px\imp\neg Bx)\vdash Ra\imp\fa x(Rx\imp Bx)\imp\neg
        Pa$
        \item $\fa x (Px\lor Bx\imp\neg Rx),\;\fa x(Sx\imp Rx)\vdash
          \fa x(Px\imp \neg Sx\lor Tx)$
          \item $\vdash\fa x(Px\land \ex yBy)\imp \ex x(Px\land Bx)$
            \item $\fa x\ex y(Px\imp Rxy)\vdash \fa x(Px\imp \ex y
              Rxy)$
              \item $\fa x(Px\imp\neg Bx)\vdash \neg\ex x(Px\land Bx)$
                \item $\ex x\fa yPxy\vdash \fa y\ex x Pxy$
  \ei
    

\item Demostrar la correctud del siguiente argumento mediante
  deducci�n natural. Usar exclusivamente la siguiente signatura:
$\{F^{(1)},V^{(1)},B^{(1)},E^{(1)},P^{(1)},L^{(1)},S^{(1)},C^{(2)}\}$
con significados {\em ser feliz,ser vampiro,ser
  bruja,emborracharse,pernoctar en cementerio, violar la ley, ser sepulturero, convivir,} respectivamente.

\emph{Un vampiro s�lo es feliz si se emborracha o pernocta en un
  cementerio. Quien pernocta en un cementerio viola la ley o es
  sepulturero. Una bruja s�lo es feliz si convive con un vampiro
  feliz, que no se emborracha y que no viola la ley. Hay brujas
  felices. Por lo tanto algunos vampiros son sepultureros.}


\item Resolver todos los ejercicios anteriores utilizando t�cticas.

\item Hallar un programa $t$ que tenga el tipo indicado:
\be
\item $\vdash t: A\imp B\imp B$
  \item $\vdash t: (A\imp B\imp C)\imp (A\imp B)\imp (A\imp C)$
    \item  $x:A\imp B\land C \vdash t: (A\imp B) \land (A\imp C)$
\item $ x: (A\imp B) \lor (A\imp C) \vdash t: A\imp B\lor C$
  \item $ x: (A\imp C) \land (B\imp C) \vdash t: A\lor B\imp C $
    \item $x: A\lor (B\land C) \vdash t: (A\lor B)\land (A\lor C)$
\ee


\item Defina la regla de tipado para el operador {\sf if}. Muestre usando su regla
que 
\[
x:Bool,y:Nat\vdash {\sf if}\; y < 2\;{\sf then\;not\;}x\;{\sf else\; iszero}\; y
\]


%\noindent {\bf Sistemas de tipos}
\item Hallar un programa $t$ que tenga el tipo indicado.
\begin{enumerate}
\item $x : P \lor (Q \land R) \vdash t : (P \lor Q) \land (P \lor R)$

\item $\vdash t : (P \to Q \to R) \to (P \to Q) \to (P \to R)$

\item $x : P \to Q \land R \vdash t: (P \to Q) \land (P \to R)$
\end{enumerate}

% \noindent {\bf T\'acticas}

% \be

\ee


\section*{Ex�menes anteriores}
%\noindent {\bf Ex\'amenes anteriores}
\begin{enumerate}

\item Mostrar las siguientes derivaciones mediante deduccion natural. Indicar en cada caso el sistema usado. En el inciso 2 la signatura es $Q^{(1)},P^{(1)},f^{(1)},g^{(1)}$
  \begin{enumerate}
  \item $(p\to q) \land (\neg r\lor q\to s)\vdash p\to \neg\neg s$ 
   \item $\ex x Qx, \;\fa x (Qx \land \ex yPy \to Qfx),\;\fa z(Qz\to Qgz)\vdash Pb \to \ex w Qfgw$.
  \end{enumerate}
\begin{comment} 
Decidir mediante deduccion natural.
\begin{itemize}
\item[a)] $\vdash_{i} A \lor B \to \lnot (\lnot A \land \lnot B)$
\item[b)] $\forall x (A \vee \neg A) \vdash \exists x A \vee \forall x \neg A $ \\
Sugerencia: Utilice el tercero excluido y el lema $ \neg \forall x \neg A \vdash_{c} \exists x A$ \\
\item[c)] $\forall x (Px \rightarrow \neg Cx), \exists x (Cx \wedge Bx) \vdash \exists x (Bx \wedge \neg Px)$ \\
Indica el sistema empleado. \\
\end{itemize}
\end{comment}

\item Considere la siguiente expresi\'on: 
\begin{center} $e =_{def} \; ({\sf not}\;({\sf iszero}\;({\sf suc}\; x))) \; 
{\sf and} \;(x < y+2))$ \end{center}

\begin{enumerate}
\item Definir la regla de tipado para el operador ${\sf and}$. 
\item Dar un contexto $\Gamma$ y un tipo ${\sf T}$ tales que se cumpla
\[
\Gamma \vdash e:{\sf T}
\]
mostrando tal derivaci\'on.
\end{enumerate}


\item Hallar un c\'odigo de prueba {\em t} de manera que se cumpla lo
siguiente:
\beqs
z: (p\to r) \lor (q\to s)\vdash t: p\land q \to r\lor s
\eeqs

\item Hallar un c\'odigo de prueba {\em t} de manera que se cumpla lo
siguiente:
\beqs
x: p\to r\; \vdash \;t: p\lor q \to r\lor q
\eeqs

\item Demuestre la correctud del siguiente argumento proposicional, mediante deducci�n natural, indicando el sistema utilizado (minimal,intuicionista o cl�sico):

\bc
{\sl Ludovico es un perico, Quique es un quelonio y Solsticio es un rat�n, Si Ludovico es un perico y Solsticio es un rat�n entonces
Eulogio no es un sapo; Tiburcio es un tlacuache o Eulogio es un sapo. Luego entonces, Tiburcio es un tlacuache.}
\ec

Defina previamente el glosario a utilizar.

\espc

\item Muestre lo siguiente mediante una derivaci�n por t�cticas:

\[
H_1:\neg p\to q\land r,\;H_2:\neg p\lor s\to\neg t,\; H_3: u\land\neg p \vdash (u\land r)\land \neg t
\]

{\sl Sugerencia: es m�s f�cil si inicia la derivaci�n con {\tt destruct $H_3$}}

\espc

\item En este ejercicio se pide construir una derivaci�n del siguiente secuente:
\[
\ex w F w \to \fa y(Gy \to Hy),\; \ex x Jx \to \ex x G x\;\vdash \fa x(Fx \land J x \to \exists zHz)
\]
Tiene la opci�n de construir una derivaci�n directa o bien una derivaci�n por t�cticas. Adicionalmente es obligatorio contestar 
por qu� prefiri� el m�todo elegido.


\espc

\item Defina la regla de tipado para el operador ${\sf entre}(e_1,e_2,e_3)$ que decide si la expresi�n aritm�tica $e_2$ est� entre $e_1$ y $e_3$, es decir, si se cumple que $e_1<e_2<e_3$. Considere la siguiente expresi�n:

\begin{center} $e =_{def} \; {\sf not\;entre}(x+2,z,y*3)$
\end{center}

\begin{enumerate}
\item[] Dar un contexto $\Gamma$ y un tipo ${\sf T}$ tales que se cumpla
\[
\Gamma \vdash e:{\sf T}
\]
mostrando tal derivaci\'on.
\end{enumerate}


\newpage




\end{enumerate}

\end{document}
 
